\documentclass[preprint]{sigplanconf}

\usepackage{amsmath}
\usepackage{amssymb}
\usepackage{natbib}
\usepackage{multirow}
\usepackage{setspace}
\usepackage{balance}
\usepackage{verbatim}
\usepackage[T1]{fontenc}  % access \textquotedbl
\usepackage{textcomp}
\usepackage[all]{xy}
\usepackage{comment}
\usepackage{datetime}
\usepackage{graphicx}
\usepackage{makecmds}
\usepackage{url}
\usepackage{multirow}
\usepackage{listings}


%include paper.fmt

%%%%%%%%%%%%%%%%%%%%%%%%%%%%%%%%%%%%%%%%%%%%%
%% GENERAL FORMATTING AND LAYOUT

% for tex2hs
\newcommand{\ignore}{}
\newcommand{\hsdef}[1]{}
\newcommand{\h}[1]{}


% temporary command to keep reusing
\newcommand{\f}{}


% from the ICFP website
\ifdefined\bibpunct\bibpunct();A{},\fi
\ifdefined\citep\let\cite=\citep\fi


% bibliography
\let\oldbibliography\bibliography
\renewcommand{\bibliography}{\oldbibliography{paper}}


% custom spacing
\newcommand{\negsmallskip}{\vspace{-\smallskipamount}}
\newcommand{\negbigskip}{\vspace{-\bigskipamount}}


%%%%%%%%%%%%%%%%%%%%%%%%%%%%%%%%%%%%%%%%%%%%%
%% LHS2TEX

% \setlength\blanklineskip{3mm}
% \setlength\mathindent{0mm}
%
%
% \renewcommand{\Varid}[1]{\mathsf{#1}}
% \renewcommand{\Conid}[1]{\mathsf{#1}}
% \renewcommand{\Varid}[1]{\mathsf{#1}}
% \newcommand{\Keyword}[1]{\text{\textbf{\textsf{#1}}}}
%
% \newcommand{\backtick}[1]{\;\mathsf{\text{\textit{\`{}}}#1\!\text{\textit{\`{}}}}\;}
%
% \newcommand{\perc}{\%
%     }
%
% % to counteract the setspace package
% \ifdefined\setstretch\renewcommand\hscodestyle{\setstretch{1}}\fi


%%%%%%%%%%%%%%%%%%%%%%%%%%%%%%%%%%%%%%%%%%%%%
%% CUSTOM COMMANDS

% \todo - write out a todo reminder, easy to grep for
\usepackage{color}
\newcommand{\todo}[1]{\textcolor{blue}{\textbf{TODO:} #1}}
\newcommand{\simon}[1]{\textcolor{green}{\textbf{SLPJ:} #1}}

\newenvironment{onepage}
	{\noindent\begin{minipage}{\linewidth}}
	{\end{minipage}}

\renewcommand{\ll}{[\![}
\newcommand{\rr}{]\!]}


%%%%%%%%%%%%%%%%%%%%%%%%%%%%%%%%%%%%%%%%%%%%%
%% EXAMPLES

\newcounter{exmp}
\setcounter{exmp}{0}

\newcommand{\noexample}{\hfill$\Box$}

% generic example (can be anything)
\newenvironment{exampleany}[1]
    {\subsubsection*{Example #1}}
    {\noexample}

% standard example
\provideenvironment{example}
    {\refstepcounter{exmp}
     \begin{exampleany}{\arabic{exmp}}}
    {\end{exampleany}}

% naming an example
\provideenvironment{examplename}[1]
    {\refstepcounter{exmp}
     \begin{exampleany}{\arabic{exmp} (#1)}}
    {\end{exampleany}}

% revisiting a previous example
\newenvironment{examplerevisit}[1]
    {\begin{exampleany}{#1 (revisited)}}
    {\end{exampleany}}

% \end{code} followed by \end{example}
\newcommand{\codeexample}{\vspace{-9mm}\par}


\lstset{
    basicstyle=\sffamily,
    columns=fullflexible,
    keepspaces=true,
    escapechar=\#,
    literate={-}{{\hbox{-}}}1
             {_}{{\hbox{\tiny\_}}}1
             {`}{{$^\backprime$}}1
             {"}{{\textquotedbl}}1
             {->}{{$\rightarrow$}}1 {<-}{{$\leftarrow$}}1
             {=>}{{$\Rightarrow$}}1
             {>=}{{$\geq$}}1
             {<=}{{$\leq$}}1
             {>>}{{>\!>}}1
             {<<}{{<\!<}}1
             {<>}{{<\hspace{-0.4mm}>}}1
             {</>}{{<\hspace{-0.4mm}/\hspace{-0.4mm}>}}1
             {<.>}{{<\hspace{-0.3mm}.\hspace{-0.3mm}>}}1
             {<\$>}{{<\hspace{-0.3mm}\$\hspace{-0.3mm}>}}1
             {<*>}{{<\hspace{-0.3mm}*\hspace{-0.3mm}>}}1
             {&&&}{{$\wedge$}}1
             {|||}{{$\vee$}}1
             {!!!}{{$\neg$}}1
             {===}{{$\equiv$}}1 }
\newcommand\lst\lstinline

\begin{document}

\setlength{\pdfpageheight}{\paperheight}
\setlength{\pdfpagewidth}{\paperwidth}

\conferenceinfo{ICFP'16}{September 18--24, 2016, Nara, Japan}
\copyrightyear{2016}
%\copyrightdata{978-1-nnnn-nnnn-n/yy/mm}
%\copyrightdoi{nnnnnnn.nnnnnnn}
% Uncomment the publication rights you want to use.
%\publicationrights{transferred}
%\publicationrights{licensed}     % this is the default
%\publicationrights{author-pays}
%\titlebanner{banner above paper title}        % These are ignored unless
%\preprintfooter{short description of paper}   % 'preprint' option specified.

\title{Non-recursive Make Considered Harmful}
\subtitle{Build Systems at Scale}

\authorinfo{Authors}
           {Affiliations}
           {Emails}
%\authorinfo{Name2}
%          {Affiliation2}
%          {Email2}

\maketitle

\begin{abstract}
Most build systems start small and simple, but over time grow into hairy monsters
that few dare to touch. As we demonstrate in this paper, there are a few issues
that cause major scalability challenges for build systems, and many pervasively
used build systems (e.g. Make) do not scale well.

This paper presents a solution to the challenges we identify. We use functional
programming to design advanced abstractions for build systems, and implement
them on top of the Shake library, which allows us to describe build rules and
dependencies. To substantiate our claims, we engineer a new build system
for the Glasgow Haskell Compiler. The result is more scalable, faster, and
spectacularly more maintainable than its Make-based predecessor.
\end{abstract}

%\category{CR-number}{subcategory}{third-level}
\keywords
build-system, compilation, Haskell

\section{Introduction}

In 1998 Peter Miller published his famously influential paper
\emph{``Recursive Make Considered Harmful''}~\cite{miller:recursive_make}.
He made a compelling case that, when designing the build system for
a large project, it is far better to ensure that Make can see
the entire dependency graph rather that a series of fragments.

Miller was right about that.  But he then went on to say ```\emph{But, but, but}'
I hear you cry.  `\emph{A single makefile is too big, it's unmaintainable,
it's too hard to write... it's just not practical}''', after which he
addresses each concern in turn\footnote{
As a historical aside, Miller points out that memory size is no
longer a problem, because
``the physical memory of modern computers exceeds 10MB''.  Indeed!}.
Here, however, he is wrong.  Using
Make for large projects really \emph{is} unmaintainable, and the rules really
\emph{are} too hard to write. \todo{\textbf{SLPJ}:
Shall we say that (Non-Recursive) Make still dominates the world of build
systems despite much time since Peter Miller's paper?}

In this paper we substantiate this claim, and offer a solution,
making the following contributions:
\begin{itemize}
\item Using the Glasgow Haskell Compiler (GHC) as a substantial exemplar,
we give concrete evidence of the fundamental lack of scalability of
Make and similar build systems (\S\ref{sec:challenges} and
\S\ref{sec:solutions}). Over the last 25 years GHC developers have implemented
no fewer than four major versions of GHC's build system; it improved each
time, but the result is still manifestly inadequate.

\item We describe Shake, an embedded domain specific language (or library)
in Haskell that directly addresses the challenges we identify
in~\S\ref{sec:shake}.
Although Shake has been introduced before~\cite{shake}, we describe here
several key features that were mentioned only in passing if at all, notably:
post-use and order-only dependencies; how to use polymorphic dependencies;
resources; and content hashes.

\item We show in some detail how Shake's built-in abstractions address
many of the scalability challenges that have been causing us such pain over
two decades (\S\ref{sec:solutions}).

\item A huge benefit of using an embedded DSL as a build system is that
we can use the facilities of the host language (Haskell) to build
abstractions on top of Shake, to fit it for the particular use case.
This sort of claim is easier to make than to substantiate; so we present
in~\S\ref{sec:abstractions} an overview of the new build system we have
developed for GHC, and the new abstractions (not part of Shake) that we built
to support it.

\item To validate our claims, we have completely re-implemented GHC's
build system, for the fifth and final time.  We replaced 10,000 lines
of (utterly incomprehensible) makefile code spread over 200 files, with
6,000 lines of (beautifully modular) Haskell code spread over 90 files.
Not only that, but the resulting system has much better behaviour in many
common use-cases, as we discuss in~\S\ref{sec:ghc}. \todo{\textbf{SDM}: would
you like to reword this to shift the focus from the line count comparison?}
\end{itemize}
None of this is fundamentally new; we review related work
in~\S\ref{section-review}.
The distinctive feature of this paper is that it is grounded in the reality
of a very large, long-lived software project.  Peter Miller would be happy.

%
%
% \todo{Note for SLPJ: there are 196 makefiles in the
% build system. This is not counting testsuite, nofib and some incidental
% makefiles, such as libraries/hoopl/paper/Makefile, which are not related to the
% build system. I expect I may have lost a file or two, or counted one which
% should not have been counted, so we can say 'around 200' makefiles. These
% makefiles contain 10416 lines and \$10764, so a little more than a dollar per
% line. Doesn't sound too impressive, but considering that there are a lot of
% comments, it's still very high. Also, if we only look into rules/*.mk we get
% 40 files, 2866 lines and \$6773, which is \$2.36 per line.
% And here is my favourite heavy-weight champion (see rules/build-prog.mk):}
%
% \begin{lstlisting}[basicstyle=\tiny]
% $1/$2/build/tmp/$$($1_$2_PROG) : $$($1_$2_$$($1_$2_PROGRAM_WAY)_HS_OBJS)
%   $$($1_$2_$$($1_$2_PROGRAM_WAY)_C_OBJS) $$($1_$2_$$($1_$2_PROGRAM_WAY)_S_OBJS)
%   $$($1_$2_OTHER_OBJS) | $$$$(dir $$$$@)/.
%
% 	$$(call cmd,$1_$2_CC) -o $$@ $$($1_$2_$$($1_$2_PROGRAM_WAY)_ALL_CC_OPTS)
% 	  $$($1_$2_$$($1_$2_PROGRAM_WAY)_ALL_LD_OPTS)
% 	  $$($1_$2_$$($1_$2_PROGRAM_WAY)_HS_OBJS) $$($1_$2_$$($1_$2_PROGRAM_WAY)_C_OBJS)
% 	  $$($1_$2_$$($1_$2_PROGRAM_WAY)_S_OBJS) $$($1_$2_OTHER_OBJS)
% 	  $$($1_$2_$$($1_$2_PROGRAM_WAY)_EXTRA_CC_OPTS)
% 	  $$(addprefix -l,$$($1_$2_EXTRA_LIBRARIES))
% \end{lstlisting}
%
% \todo{The first line contains \$42 and the second one the absolute record \$64!}
%
% \todo{The new build system contains 91 Haskell files, 5753 lines, and just
% \$600.}
%
% A build system is a critical component of any software project. It's the
% coal-face at which developers spend their days, so must be fast to run, produce
% the outputs correctly, and provide all the features needed. Despite the
% importance of writing good build systems, such endeavours remain a black art.
% The build system for the Glasgow Haskell Compiler~(GHC)~\cite{ghc} has been
% rewritten three times using \make{}~\cite{make}, and while lessons have been
% learnt each time, the end result remains frustratingly complex. As an example:
%
% \begin{lstlisting}
% $$(foreach dep,$$($1_$2_DEP_COMPONENT_IDS),\
%     $$$$($$(dep)_dist-$(if $(filter 0,$3),boot,install)\
%         _PROGRAM_DEP_LIB))
% \end{lstlisting}
%
% Taking a look at \lst"$$($1_$2_DEP_COMPONENT_IDS)" -- this constructs a variable
% name using the variables \lst"$1" and \lst"$2" (themselves macro arguments),
% then resolves this variable. It then takes the value of that variable and
% resolves it. The end result is almost impossible to predict. The definition site
% of any individual symbol is hard to track down. Many are defined, redefined and
% conditionally defined. While the first line is confusing, the second is
% significantly more so, with 4 indirections of a computed variable.
%
% Were this the only example in the build system, we might be able to solve it
% with lots of documentation. But alas, this snippet is the rule, not the exception.
%
% In this paper we show a better approach to designing large build systems. Our
% approach is two pronged:
%
% \begin{itemize}
% \item  We use the latest innovations in build systems (specifically Shake,
% \cite{shake}), combined with the abstraction power of functional programming (in
% our case, Haskell \cite{haskell}), to remove much of the \textit{accidental
% complexity} from the build system. By using sensible primitives that compose
% nicely we are able to remove much of the heroism required from build system
% maintenance.
% \item We identify suitable abstractions for writing large build systems
% \S\ref{sec:abstractions}. In particular we describe an embedded domain-specific
% language (EDSL) for capturing much of the configuration of the build system
% (which optimisation flags are required to which tools in which modes), cleanly
% separating this configuration from the rest of the build system.
% \end{itemize}
%
% We have used our approach to write a new GHC build system \S\ref{sec:ghc}. The
% new build system is about half the length of the previous one (5,824 lines as
% opposed to 10,952). \todo{The new system has reasonable chunks of the old system
% as comments. Should check non-comment and non-blank lines.} More satisfyingly,
% the new build system remains intelligible, and hopefully will not require
% further rewrites.
%
% % SimonM: the new build system is still missing some functionality,
% % correct?  So how can we get an accurate comparison of size?
%
% %  \item Motivating examples of mission-critical build systems, perhaps, from
% %  Standard Chartered and/or Facebook?
%
% ~\\
% \textbf{Contritubions}:
% \begin{itemize}
%   \item A convincing demonstration for the need of abstractions,
%   their benefits for maintenance and use.
%   \item Design pattern for a DSL and escape hatch (Neil)
%   \item Particular set of abstractions for GHC.
%   \item "New" abstractions in Shake: oracles, resources, unchanging files.
%   \item How Shake solves make's problems
%   \item Non-recursive Make Considered Harmful.
%   \item Version 5 of GHC build system; the paper is informed by
%   experience of a truly large scale build system (used to
%   have 800 files with N generated).
%   \item Flag provenance tracking and simpler provenance as no
%   generated files.
% \end{itemize}

\section{Challenges of large-scale build systems\label{sec:challenges}}

Many existing build systems work just fine for small projects, or
projects that follow a common pattern.  For example, building a single
executable or library from single-language source files is a
well-supported common case covered by virtually all build systems.  A
lot of projects fit into this category, and so never run into the
limits of existing build systems.

Things start to get hairy in big projects, when complexities such as these
show up:
\begin{itemize}
\item The project has a large number of components (libraries or
  executables), using multiple languages, with complex
  interdependencies.  Executables depending on libraries, for example,
  or tools generate inputs for other parts of the build system.
\item Many components follow similar patterns, so there is a need for
  abstractions that allow common functionality to be shared between
  different parts of the build system.
\item Parts of the build system are not static and are generated,
  perhaps by running tools that are also built by the same build
  system.
\item The build system has complex configuration, with aspects of its
  behaviour being controlled by a variety of different sources:
  automatic platform-specific configuration, command-line options,
  configuration files, and so on.
\end{itemize}
The GHC build system includes all of the above. To illustrate the
consequences, here is one rule in GHC's current build system, which
is based on \make{}:
\begin{lstlisting}[basicstyle=\footnotesize\sffamily,escapeinside={(*}{*)}]
$1/$2/build/%.$$($3_osuf) : \
    $1/$4/%.hs $$(LAX_DEPS_FOLLOW) \
    $$$$($1_$2_HC_DEP) $$($1_$2_PKGDATA_DEP)
  $$(call cmd,$1_$2_HC) $$($1_$2_$3_ALL_HC_OPTS) \
    -c $$< -o $$@ \
    $$(if $$(findstring YES,$$($1_$2_DYNAMIC_TOO)), \
       -dyno $$(addsuffix .$$(dyn_osuf),$$(basename $$@)) )
  $$(call ohi(*-*)sanity(*-*)check,$1,$2,$3,$1/$2/build/$$*)
\end{lstlisting}
Yes, there are four dollar signs in a row!  If it wasn't so tragic,
impenetrable code like this would be hilarious.  This kind of thing
goes on for \emph{thousands} of lines.  The result is a nighmare to
maintain and modify.

Perhaps GHC's implementors just aren't very clever?  Maybe so, but
they did at least try hard. The current build system is the
fourth major iteration, as we struggled to find good solutions to the
problems that arose:

\begin{itemize}
\item The first incarnation of the build system used \emph{jmake}, a
  system inspired by the X Consortium's \emph{imake}.  This was
  essentially ordinary \make{} together with the C preprocessor to
  allow the use of macros in \lst"Makefile"s.  The macro layer partly
  solves the problem of needing to share functionality between
  different parts of the build system.

\item The C preprocessor was somewhat painful to use with \make{}.
  The syntax was difficult to get right, and the need to constantly
  \lst"make Makefiles" when working on the build system was tiresome.  GNU
  \make{} came along which had built-in support for \lst"include" files and
  other features, and this meant that we could do away with the C
  preprocessor.  We rewrote the build system to use GNU \make{}, and
  simulated macros with include files (GNU \make{} didn't have macros at
  the time).

\item Then there were several large changes to the GHC build system
  that didn't amount to complete rewrites.  First we make the build
  system able to build multiple bootstrap \emph{stages} in a single
  build tree; that is, build the compiler (stage 1) and then build the
  compiler again using the stage 1 compiler (stage 2).  Previously
  this had required two separte builds.

  Around this time the library ecosystem of Haskell was exploding, and
  the Cabal build system for Haskell libraries emerged.  We wanted to
  integrate Haskell libraries with Cabal build systems into the GHC
  build, but without duplicating the Cabal metadata or build logic for
  those packages.  So we integrated our build system with Cabal,

\item The build system in its current form had grown unweildy, and had
  many idiosyncrasies.  The root of many of the problems was that the
  build system was constructed as a set of Makefiles that recursively
  invoked each other.  Recursive \make{} is considered harmful
  for very good reasons~\cite{miller:recursive_make}; it is not possible to
  accurately track dependencies when the build system is constructed
  of separate components that invoke each other.

  So in the next rewrite of the build system we made it non-recursive.
  However, in doing so we stretched GNU \make{} to its absolute
  limits, as the example above illustrates.  Every part of the rule is
  there for a good reason, but the cumulative effect of solving all
  the problems that arise in a complex build system using a tool that
  is not equipped to cope with it, is impenetrable.
\end{itemize}
\noindent
By carefully employing a set of
idioms\footnote{\url{https://ghc.haskell.org/trac/ghc/wiki/Building/Architecture}}
(such as the prefix idiom above), we managed to construct a non-recursive build
system for GHC in GNU \make{}.  It works surprisingly well, and by
virtue of being non-recursive it tracks dependencies accurately;
but, it is almost impossible to understand and maintain.  So while
it is \textit{possible} to develop large scale systems in \make{}, it is
clear that we have gone far beyond the scalability limits of \make{}.

% \begin{enumerate}
% \item The programming language is complex. This macro expands through multiple levels, so \lst"$$$$($1_$2_HC_DEP)" looks up in the environment two stages later.
% \item The dependency on \lst"$1_$2_PKGDATA_DEP" is only to reduce parallelism, to ensure too many commands do not run in parallel.
% \end{enumerate}

% \subsection{The GHC build system}
% 
% Build systems need to evolve quickly with the projects they support. While it is
% \textit{possible} to develop large scale systems in \make, it is not
% \textit{pleasant} -- resulting in heroic efforts where straight-forward
% simplicity should be preferred.
% 
% The GHC build system is a complex multi-language build system. The system is a
% bootstrapping compiler, where the GHC compiler is built using a system compiler,
% then recompiled using built compiler and recompiled system libraries. This
% pattern naturally contains repeated patterns, but capturing them in \make{} is
% hard.
% 
% We focus on the GHC build system for two reasons. Firstly, it is the coal-face
% at which many of us work, so improvements to it will bring real improvements to
% our daily development. Secondly, the GHC build system has many complex features
% that test the limits of existing build systems:
% 
% \begin{itemize}
% \item GHC is cross-language, including large amounts of both C and Haskell code.
% It also generates user manuals from docbook, which can be viewed as another
% language with unique build/dependency patterns.
% \item GHC is a bootstrapping compiler with stages. It first builds a compiler
% using the system compiler, then uses that new compiler
% \item The GHC build system necessarily integrates with other build systems,
% including \make{} (for building libgmp) and \cabal{} (for building/registering
% Haskell libraries).
% \item It generates files, e.g. \texttt{compiler/stage1/ghc\_boot\_platform.h}
% that \texttt{\#define}'s various platform-specific constants used throughout the
% build system.
% \item It is cross-platform, working on Windows, Linux, Mac, iOS, Android,
% Solaris, BSD flavours etc.
% \item The system is in constant flux as new features are added to the system.
% \end{itemize}
% 
% Writing such a build system remains a challenging engineering undertaking, but
% one we now hope not to have to repeat. This is the fifth version, and interestingly, perhaps the most user friendly. Applying all the guidelines in Recursive Make Considered Harmful we have gone through the following stages.
% 
% \begin{enumerate}
%   \item jmake - cpp + make clone of X11 imake
%   \item GNU make - drop cpp, recursive make, manual stages
%   \item GNU make - no manual stages, cabal integration, first
%    use of macros
%    \item GNU make - current iteration, non recursive, cite Recursive
%      Make Considered Harmful here, extensive use of macros,
%      building abstractions in make
% \end{enumerate}

\section{Background about Shake\label{sec:shake}}

The Shake build system was introduced by \citet{shake}. In this section we briefly recap the key ideas of Shake from that paper. Next we discuss some of the inovations in Shake since that time. All are generally useful and were in wide use before we started investigating the GHC build system. Much of this section can be considered the bits needed to take us from the theory to the practice.

\begin{figure}
\begin{lstlisting}
newtype Rule a = ... deriving (Monoid, Functor, Applicative, Monad)
newtype Action a = ... deriving (Functor, Applicative, Monad, MonadIO)

shake :: ShakeOptions -> Rules () -> IO ()
action :: Action a -> Rules ()

type ShakeValue a = (Show a, Typeable a, Eq a, Hashable a, Binary a, NFData a)
data EqualCost = EqualCheap | EqualExpensive | NotEqual

class (ShakeValue key, ShakeValue value) => Rule key value where
    storedValue :: Rule key value => ShakeOptions -> key -> IO (Maybe value)
    equalValue :: Rule key value => ShakeOptions -> key -> value -> value -> EqualCost

rule :: Rule key value => (key -> Maybe (Action value)) -> Rules ()
apply :: Rule key value => [key] -> Action [value]
\end{lstlisting}
\todo{Include everything we use in the paper}
\caption{Shake generic API}
\end{figure}

\begin{figure}
\begin{lstlisting}
(?>) :: (FilePath -> Bool) -> (FilePath -> Action ()) -> Rules ()
want :: [FilePath] -> Rules ()
need :: [FilePath] -> Action ()
needed :: [FilePath] -> Action ()
\end{lstlisting}
\caption{Shake file-specific API}
\end{figure}

\subsection{Introduction}

The two key types in Shake are \lst"Rules" and \lst"Action". The \lst"Rules" represents the list of things Shake knows how to build, and each rule has an associated \lst"Action" which is the list of actions to build. The \lst"Rules" is a monoid, allowing two sets of rules to be joined to form a new set of rules. The action is a \lst"MonadIO" allowing actual actions to be run. As an example:

\todo{start with main, complete build system}

\begin{lstlisting}
(\x -> takeExtension x == ".o") ?> \out -> do
    let src = replaceExtension x ".c"
    let mk = replaceExtension x ".m"
    need [src]
    cmd "gcc -o" out "-M" mk "-c" src
    neededMakefileDepends mk
\end{lstlisting}

There the whole block is a \lst"Rules", while the 5 indented lines are the \lst"Action". The action says what to match, namely all files with a \lst".o" extension - object files. The action gets run with \lst"out" bound to the output name, e.g. \lst"Foo.o". It then computes \lst"src" (e.g. \lst"Foo.c") and \lst"mk" (e.g. \lst"Foo.m"). It requires the \lst".c" file to exist and builds it if it is missing, then compiles, and finally introduces a dependency on all the files listed in the \lst".m" file.

There is a database which tracks this stuff. More on monadic dependencies. When will this example rerun. If the output file changes it will rerun.

\subsection{Polymorphic dependencies}

Shake has polymorphic dependencies -- any key can produce any value by defining custom type classes. However, it turns out that one use of this mechanism is sufficiently common that it can be given a wrapping. We define an \textit{Oracle} to be a key/value mapping like a rule with the restrictions that 1) there is only one action associated with all members of the type; 2) the oracles are rerun in every execution and are never cached. While restrictive, this pattern captures a number of invariants, e.g. checking external program version numbers have not changed. As an example, we can define:

Do oracle example by hand, then do it sugared up with oracles. Forward reference to config files. Put the long example in a figure.

\begin{lstlisting}
newtype GccVersion = GccVersion ()
    deriving (Show,Typeable,Eq,Hashable,Binary,NFData)

rules = do
    addOracle $ \GccVersion{} ->
        fromStdout <$> cmd "gcc --version" :: Action String
    ... ?> \out -> do
        askOracle $ GccVersion () :: String
\end{lstlisting}

Here we have tracked the GccVersion we are relying on, so if the version of gcc changes anything will rerun.

While offering no real power, by simplifying the common case they provide easy usage. The implementation of Oracles is simple:

\begin{lstlisting}
newtype OracleQ question = OracleQ question
    deriving (Show,Typeable,Eq,Hashable,Binary,NFData)
newtype OracleA answer = OracleA {fromOracleA :: answer}
    deriving (Show,Typeable,Eq,Hashable,Binary,NFData)

instance (ShakeValue q, ShakeValue a) =>
    Rule (OracleQ q) (OracleA a) where
        storedValue _ _ = return Nothing

addOracle :: (ShakeValue q, ShakeValue a)
          => (q -> Action a) -> Rules ()
addOracle act = void $ rule $
    \(OracleQ q) -> Just $ OracleA <$> act q

askOracle :: (ShakeValue q, ShakeValue a)
          => q -> Action a
askOracle question =
    fmap fromOracleA $ apply1 $ OracleQ question
\end{lstlisting}

\subsection{Unchanging files}

Add a dependency diagram.

Dodgy - can make already do this? It can deal with not updating.

Of course, oracles rerun in each section. One feature of Shake that turns out to be key to making oracles work is the fact that if oracles rerun and return an equal value they do not make anything that depended on them dirty. While observed as useful for generated files in the original paper, it turns out to be key for pervasive tracking of the detials that in many build systems go missed.

\subsection{Shared cache}

The abstraction.

\subsection{Order-only dependencies}

One dependency feature missing from the original Shake paper was order-only dependencies. An order-only dependency is one that must be built before continuing, but if the order-only dependency changes this rule does not need to rerun. A legitimate use of such a dependency is that an action might read any one of two files, and after the fact, can report which files it actually depended upon. Usually then a subset of these files will be added as explicit dependencies afterwards.

This pattern can be modelled in Shake using a rule whose key type is \lst"()" -- one that always compares equal, and the user defining a tag for the closure required, so it can be stored in the database. While workable, the pattern is not particularly reusable, so we instead provide a function that directly resets the dependency state.

Is this used? Add an example.

\subsection{Resources}

When you run -j10 (shakeThreads=10) you are asking the build system to limit computation so it uses no more than ten CPU resources at a time. The CPU is certainly a precious resource, but there are other resource limitations a build system may need to obey:

\begin{enumerate}
\item Some APIs are global in nature, if you run two programs that access the Excel API at the same time things start to fail.
\item Many people have large numbers of CPUs, but only one slow rotating hard drive. If you run ten hard-drive thrashing linkers simultaneously the computer is likely to grind to a halt.
\item Some proprietary software requires licenses, a fixed number of which can be purchased and managed using a license manager. As an example, the Kansas Lava team only have access to 48 licenses for modelsim.
\end{enumerate}

I know of three approaches used by other build systems to obey resource constraints:

\begin{enumerate}
\item Limit the number of CPUs to hit your target - for example, the Lava build system could cap the number of CPUs to the number of licenses. People with 24 CPUs might ask the build system to use only 8, so the linkers do not make their machines unusable (and even then, a link heavy rebuild may still harm interactive performance). This solution wastes CPU resources, leaving CPUs that could be building your code idling.
\item Add locks to suspend jobs that are competing for the shared resource. For example any rule using Excel could take the Excel lock, either a mutex/MVar in some build systems, or creating a file to serve as the lock in make based build systems. Locking can be made to work, but is tricky if you have to fake locks using the file system, and still squanders CPU resources - instead of blocking the CPU should be running another rule.
\item Use dependencies in sequence to ensure that the items running in parallel are serialised.
\end{enumerate}

In Shake the Resource type represents a finite resource, which multiple build rules can use. Resource values are created with newResource and used with withResource. As an example, only one set of calls to the Excel API can occur at one time, therefore Excel is a finite resource of quantity 1. You can write:

\begin{lstlisting}
shake shakeOptions{shakeThreads=2} $ do
    want ["a.xls","b.xls"]
    excel <- newResource "Excel" 1
    "*.xls" *> \out ->
        withResource excel 1 $
            system' "excel" [out,...]
\end{lstlisting}

Now we will never run two copies of excel simultaneously. Moreover, it will never block waiting for excel if there are other rules that could be run.

Fwd ref to ghc-pkg.

\subsection{Modification tracking}

Build systems run actions on files, skipping the actions if the files have not changed. An important part of that process involves determining if a file has changed. The Make build system uses modification times to impose an ordering on files, but more modern build systems tend to use the modification time as a proxy for the file contents, where any change indicates the contents have changed (e.g. Shake, Ninja). The alternative approach is to compute a hash/digest of the file contents (e.g. SCons, Redo). As of version 0.13, Shake supports both methods, along with three combinations of them - in this post I'll go through the alternatives, and their advantages/disadvantages.

Modification times rely on the file-system updating a timestamp whenever the file contents are written. Modification time is cheap to query. Saving a file afresh will cause the modification time to change, even if the contents do not - as a result touch causes rebuilds. Unfortunately, working with git branches sometimes modifies a file but leaves it with the same contents, which can result in unnecessary rebuilds. We can view the modification time as a surjective function from modification time to file contents.

File digests are computed from the file contents, and accurately reflect if the file contents have changed. There is a remote risk that the file will change without its digest changing, but unless your build system users are actively hostile attackers, that is unlikely. The disadvantage of digests is that they are expensive to compute, requiring a full scan of the file. In particular, after every rule finishes it must scan the file it just built, and on startup the build system must scan all the files. Scanning all the files can cause empty rebuilds to take minutes.

To get the best of both worlds Shake can store the modification time, file size and hash of the contents. After producing a file all information is stored. When checking, first the modification time is checked, and if it matches, the contents have not changed. If the modification time has changed, and the size has changed, then the contents does not match. Only in the case where the modification time has changed but the size has not do we have to compute the actual hash. If after doing that the contents are equal we store a new modification time, so that future checks will be fast. If the contents have changed then the file will likely be rebuilt, and thus will be written afresh with the new hash and modification time.

While a signficant optimisation over always checking file hashes, for certain large files the computation of a hash can still be quite expensive (although almost always cheaper than producing the file). To reduce that problem, Shake has a mode that only digests for source files that are not written by the build system. Generated files (e.g. compiled binaries) tend to be large (expensive to compute digests) and not edited (rarely end up the same), so a poor candidate for digests. The file size check means this restriction is unlikely to make a difference when checking all files, but may have some limited impact when building.

Describe the git pattern. Remove the description of configuration.

\subsection{Lint checks}

\S? of \cite{shake} postulates a number of invariants, and using the \prog{FSATrace} program these can now be checked at runtime. Shake has a \texttt{--lint} flag which also checks the current working directory does not change (a common mistake in a global build system, as the current working directory is a shared resource). It also checks that files do not change after they have been depended upon, that running a fresh build after a build completes will have no further effect, and that.

We also have a function \lst"needed", rather than \lst"need" that asserts that the result of performing \lst"need" does not cause the file to change. This is typically required to depend on files that have already been used, e.g. the header files have already been scanned, so if they were to build afresh that would be an error.

Most useful one is if two people whack the same output. Other thing is where doing all .hs files in a dir, and then you generate one. Should this go somewhere else? Perhaps in S5.

\subsection{Design pattern: DSL + escape hatch}

\todo{Probably belongs elsewhere, but don't want to conflict}

As we develop more Shake-based build systems, patterns have started to emerge. Typically 90\% of a build system can be captured in some simple DSL, taking advantage of conventions, and 10\% cannot. As an example, a large build system might build 100 C++ libraries, each with similar flags and file layout, but taking the source files from different directories. It may also minify a Javascript file, and build an installer -- both one-off tasks. Using a fully-powerful system such as Shake, it is possible to engineer robust abstractions, and then define the majority of the build system using only these abstractions. The end result is that most edits to the build system involve only the DSL, and can be performed by a large number of individuals.

After dealing with the DSL, there are usually a few pieces left over, and these can be implemented in Shake, as normal. Thanks to the power of Shake you can interpret the DSL and combine it with the custom pieces. In our experience by providing an \textit{escape hatch} where fully-powerful code an be expressed it removes the temptation to shoehorn more advanced features in the DSL, and thus avoids turning it into an ad-hoc scripting language. Instead, should enough pieces be required in the escape hatch, they can be abstracted in the traditional ways - perhaps even combining two DSLs in one build system.

\section{Solutions\label{sec:challenges}}

The GHC build system stumbles into a number of nasty corners of \make{}. In this section we reflect on the challenges, and how they can be solved generically in our new system. We very much consider these \textit{unnecessary} complexities -- they are not consequences of our problem domain, merely weaknesses of \make{} on large projects. In general the problems can be divided into those due to the \make{} language (macros, variables etc.) and those due to the \make{} dependency features (lack of expressive dependencies). Consequently, we tackle these problems using functional programming for the language level, and Shake for the dependency level. After tackling these unnecessary complexities, we show how to construct a large build system \S\ref{sec:abstractions}.

\subsection{Programming model}

\make{}'s program state involves a global namespace of string variables which are spliced into the program. This model naturally causes challenges:

\begin{itemize}
\item Since all variables live in a single global namespace, there is limited encapsulation and implementation hiding. While macros introduce some level of scope, their use leads to different problems.
\item Excessive freedom, implementing \lst"$$$$foo" to do multiple lookups. Such code is much like \lst"eval" in a scripting language, but it is unlikely any scripting language author would encourage such a level of indirection. Alas, without real data structures (associative maps, arrays) such tricks are necessary.
\item Since the text is spliced into the resulting Makefile to be interpreted, certain characters can pose lots of problems -- notably space (which splits lexemes which might have been intended to be joined) and `:' (which declares rules, but often tends to be a Windows drive letter gone astray). Much like command line quoting, with sufficient care such issues can be avoided -- but it does require constant care.
\item A string variable, such as \lst"$1_$2_$3_HC_OPTS", can be manipulated in many different files and it is very difficult to track the provenance of a specific command line argument that is eventually passed to a command line tool.
\end{itemize}

The solution to the issue of encapsulation is a properly block-scoped language with separate units of implementation hiding, e.g. modules in Haskell. The solution to excessive freedom is types (not necessarily static). The solution to splicing is to separate values and program text, much like most programming languages do. By using Haskell, or indeed any other modern programming language, most of these issues are solve.

An alternative solution would be to avoid putting complexity into the Makefile, and place it in a generator of a Makefile (much like the Ninja build system does). Alas, such a solution runs into problems if there is a dynamic call graph, as per \S?.

At this point it is worth remarking that many alternative build systems do indeed offer real programming language embedding, e.g. SCons (Python). We address the specific limitations that drove us away from them in later sections, and discuss all alternatives in \S?.

\subsection{Pattern/rule language}

A build system builds some output files from some input files. The fundamental unit of work in \make{} (and also Shake) is a rule that produces some output by running commands on some input. Consider the following pattern rule:

\begin{lstlisting}
%.o : %.hs
    ghc $HC_OPTS $<
\end{lstlisting}

\noindent It tells \make{} that object files \lst"*.o" can be produced from Haskell source files \lst"*.hs" by compiling them with \lst"ghc" command invoked with \lst"HC_OPTS" arguments. The notation is terse and works well for this simple case. Unfortunately, this simple formulation does not support numerous important features, for example:

\begin{itemize}
\item What if we want the rule to match \lst"foo.o" and \lst"bar.o", but not \lst"baz.o"? It is impossible to do any non-trivial computation -- we are forced to rely on patterns whose expressive power is limited.
\item What if \lst"HC_OPTS" depends on the file being compiled? At the moment it is a fixed global variable.
\end{itemize}

\newsavebox{\exampleCode}
\begin{lrbox}{\exampleCode}
\begin{minipage}[t]{\columnwidth}
Here is the code in full glory. We cannot go into details due to lack of
space, but would like to point out that there are \$57 in three lines!
\begin{lstlisting}[basicstyle=\footnotesize\sffamily,escapeinside={(*}{*)}]
$1/$2/build/%.$$($3_osuf) : $1/$4/%.hs $$(LAX_DEPS_FOLLOW) \
    $$$$($1_$2_HC_DEP) $$($1_$2_PKGDATA_DEP)
  $$(call cmd,$1_$2_HC) $$($1_$2_$3_ALL_HC_OPTS) -c $$< -o \
    $$@ $$(if $$(findstring YES,$$($1_$2_DYNAMIC_TOO)),-dyno \
    $$(addsuffix .$$(dyn_osuf),$$(basename $$@)))
  $$(call ohi(*-*)sanity(*-*)check,$1,$2,$3,$1/$2/build/$$*)
\end{lstlisting}
\end{minipage}
\end{lrbox}

The standard approach to overcome some of \make{}'s limitations is to use
\emph{macros}. They provide some additional flexibility, but quickly become difficult to follow. As an example, here is a \emph{significantly simplified} snippet\footnote{\usebox{\exampleCode}} from the GHC build system:

\begin{lstlisting}
$1/$2/build/%.o : $1/$4/%.hs $$$$($1_$2_HC)
    $1_$2_HC $$($1_$2_$3_HC_OPTS) -c $$< -o $$@
\end{lstlisting}

\noindent As before, the rule is responsible for compiling a Haskell source
file into an object file. Arguments of the macro (\lst"$1" to \lst"$4") provide
additional information about the current build target so that we can pick the
right Haskell compiler \lst"$1_$2_HC" and run it with appropriate command
line arguments \lst"$1_$2_$3_HC_OPTS".

Using Shake the simplest variant looks slightly more complex as it has had Haskell syntax inserted:

\begin{lstlisting}
"*.o" %> \out -> do
    let hs = out -<.> ".hs"
    need [hs]
    cmd "ghc" hc_opts hs
\end{lstlisting}

However, as the complexity grows, so the build system scales properly. Making \lst"hc_opts" depend on the Haskell file requires the small and obvious change of:

\begin{lstlisting}
    cmd "ghc" (hc_opts hs) hs
\end{lstlisting}

To use richer pattern matching we can drop down to a lower level Shake operation. In Shake the definition of is \lst"%>" is itself defined in terms of \lst"?>", as:

\begin{verbatim}
pat %> act = (pat ?==) ?> act
\end{verbatim}

Namely the wildcard pattern matching is just a special case, so we can use an arbitrary predicate to exert more precise control over where the match occurs. In both cases the generality of higher-order functions solves the problem.

\subsection{Multiple outputs}

\todo{Make supports multiple outputs. We need to be careful about what the rules are. Is this a genuine weakness or not?}

In the rule above, GHC actually produces two files: \lst"*.o" and \lst"*.hi". How do we express this?

Mostly by pretending the .hi file depends on the .o file, which is (mostly) true. But only really because we always touch the .o file.

Using Shake we can express this pattern directly (it actually turns out to be a consequence of the more powerful dependency model).

\subsection{Reducing concurrency}

Usually we want the build system to run as many steps as possible in parallel at
any point. However, for some actions, executing multiple steps in parallel can
cause problems. As one example, in GHC packages need to be registered by
invoking the \prog{ghc-pkg} utility. This utility mutates the global state
(package database) and hence at most one package can be registered at a time, or
the database is corrupted. As another example, when using Microsoft Excel via
the scripting API a hidden Excel process is spawned and used for communication.
Running two programs against the scripting API can result in actions applying to
the wrong instance.

In \make{}, the way to reduce concurrency is to introduce fake \emph{concurrency
reduction} dependencies. In the GHC build system there are 25 packages that
require registration, and they depend on each other in a chain, to ensure one
completes before the next starts. This pattern works, but is inelegant.

There are really three problems. By over-sequentialising things we limit other
parallelism that may be available in the system. We cannot say that all pieces
should be run single-threaded, we must over-constrain to give a precise order in
which they can build. If one of the later pieces is available to build, we may
stop it by waiting for another. Finally, it's easy to miss such dependencies as
they are not centralised. By accidentally turning the distributed line into a
tree we can end up with surprising failures.

The solution is Shake is to use the \emph{resources} feature. This feature limits parallelism by declaring that a certain task requires some quantity of a global resource. For example:

\begin{lstlisting}[basicstyle=\ttfamily]
packageDb <- newResource "package-db" 1
action $ withResource packageDb 1 $ cmd "ghc-pkg" ...
\end{lstlisting}

This snippet declares a global resource named \lst"packageDb" with quantity 1,
then calls \lst"withResource" asking for a single quantity before running the
program \prog{ghc-pkg}. Provided all \prog{ghc-pkg} calls are suitably wrapped,
we will never run two instances in parallel. Furthermore, thanks to the
availability of functions, we can abstract a function that executes
\prog{ghc-pkg} and applies the resource automatically, see
\S\ref{sec:abstraction} for how we do that.

Note that build systems already schedule tasks to satisfy the implicit thread
resource, so having user defined resources is an obvious generalisation.

The only other build system we are aware of that contains resources is
\prog{Ninja}, which calls them pools.

\subsection{Dynamic dependencies}

The \make{} approach builds a dependency graph and executes it in an order satifying the dependencies. This approach is shared by most build systems. However, for large build systems, the dependency graph is not static and is not known in advance. As an example in GHC:

\begin{itemize}
  \item File \texttt{compiler/prelude/primops.txt.pp} describes GHC's primitive
  operations and types. It uses C preprocessor directives like
  \texttt{\#include} and \texttt{\#if} and therefore needs to be preprocessed.
  The result goes into \texttt{compiler/stage1/build/primops.txt}. If one of the
  included files changes the result must be rebuilt. Note: we do not know in
  advance which files are included, so we cannot depend on them statically,
  that is why such dependencies are called dynamic.
  \item File \texttt{primops.txt} is processed by \texttt{genprimopcode} utility
  to generate \texttt{primop-out-of-line.hs-incl}. \texttt{genprimopcode} itself
  needs to be built. It is a Haskell program, so it may contain \texttt{import}
  directives, which also give rise to dynamic dependencies.
  \item A dozen of \texttt{*.hs-incl} files are included (by a C preprocessor)
  into \texttt{compiler/prelude/PrimOp.hs} and also need to be tracked
  dynamically.
  \item \texttt{compiler/prelude/PrimOp.hs} is imported in many other source
  files, and participates in the rest of the build process in a more or less
  standard way.
\end{itemize}

In essence, the problem is that \lst"import" and \lst"include" statements induce dependencies, but these dependency statements are themselves generated by the build system, so the build system cannot know the dependency graph in advance.

In \make{} there are workarounds for this problem, but they are all ugly. One is to over-approximate the possible set of dependencies, but in many ways this relies on the contents of the generator, and is thus brittle to generator changes. Another approach is to divide the build into phases, where each phase is static, and the phases are run sequentially. Such a global phase ordering complicates the build and breaks compositionality.

\todo{How does the existing GHC build system solve this?}

In Shake the problem of dynamic dependencies is solved by providing the \lst"need" function, which can introduce new dependencies \emph{after observing the results of previous dependencies}. There is no graph constructed in advance.  After generating \lst"primop-out-of-line.hs-incl" we read the file to see what it imports, requiring no conservative approximations or phases.

While the existence of \lst"need" fundamentally alters the expressive power of Shake, for users it is merely a relaxation of an unnecessary rule (that \lst"need" must be the in a rule), so just makes things easier. At the build system level it requires many changes, the lack of a static dependency graph, termination checking as it executes, the ability to suspend partially executed rules etc. Fortunately, as users we do not need to care.

\textit{Aside/Footnote}: It has been suggested that the dependency mechanism in Shake should rightly be called Monadic dependencies to contrast with the Applicative dependencies in Make. We agree. In an applicative computation the structure cannot depend on the values flowing through container. In a monadic computation the structure can depend on the values.

Note that Redo is the only other clearly monadic build system, while it can be encoded in SCons using the dependency rules.

\subsection{Splitting dependencies}

There are many instances where an action depends on a subset of a file. As a compelling example in GHC, there are many indiviudal configuration flags provided in a file \lst"system.config", which reads:

\todo{include an example}

\begin{lstlisting}
system-ghc = /usr/bin/ghc
system-gcc = /usr/bin/gcc
\end{lstlisting}

When executing a command we typically depend on only a handful of lines in this large configuration file. However, using simple file dependencies, we have to depend on the whole file. As a result, we end up rebuilding more than is necessary.

One solution is to create a single file for each setting, but using \prog{configure} that is challenging. Since a single file is being generated, the next best solution is to chop up the file into multiple small files, to provide accurate fine-grained dependencies.

Taking this approach doesn't work in \make{}. We can produce individual files, but the individual files must depend on the large file, so they update appropriately. In the rule to generate the fragments, if we always generate the new file, then we do not break the dependencies -- everything still has an ultimate dependency on the whole file. If in a rule we chose to avoid generating the small file then the file always looks dirty (since its timestamp is older than the big file, which \make{} considers to signify dirty).

Fortunately, Shake has an \emph{unchanging files} feature which solves this problem. Using the same approach in Shake we can simply avoid writing out the small file if it has not changed. Shake considers a file to be rebuilt if the rule runs successfully, but only considers the file \emph{changed} if its value changed. As a result, Shake can start down a build tree and then later avoid that build tree. This feature is built into the core of Shake, and means that, unlike \make{}, we can split a single dependency into fine-grained dependencies successfully.

The results of such a change can be dramatic -- on certain real build systems we regularly see build times of a couple of seconds, which would have been minutes without this feature. This feature is also available in Tup and Ninja, although in both cases requires explicitly enabling per rule (using \texttt{stat} in Ninja, and \texttt{\^} in Tup).

\subsection{Everything is a file}

Once fine grained dependencies are available, they can quickly become pervasive. Using a build system based on filenames as keys and files as values we effectively reach the global store problem we criticised in \S? (particularly if we consider dyanmic dependencies from \S? to be \lst"$" dereferencing!). While not critical, Shake supports an Oracle feature, to allow us to make the keys and values arbitrary Haskell types, and store them in the main Shake database. Thanks to such a technique we can have an oracle per rule we create to track changing information, without necessarily doubling the number of files present in the build system. As an example:

\begin{lstlisting}
newtype ConfigKey = ConfigKey String
    deriving (Show,Typeable,Eq,Hashable,Binary,NFData)
rules = do
    addOracle $ \(ConfigKey x) -> do
        src <- readFile' "system.config"
        Map.lookup (parseFile src) x

    ... askOracle (ConfigKey x) ...
\end{lstlisting}

Here we define our own rule of type \lst"ConfigKey" which reads from the configuration and stores just that one field. Since this also participates in unchanging rules we automatically get rebuild avoidance if the value does not change.

\subsubsection{Real code}

Much of a build system is calling out to programs that execute real code, e.g. compilers. But there is some stuff that requires custom code, for example splitting command line arguments that exceed a certain size. In the existing build system we use xargs, which alas doesn't work consistently between different OS versions, and does both more than we want and less than we want. Fortunately, with Haskell at our disposal, we can write:

\begin{lstlisting}
-- | @chunksOfSize size strings@ splits a given list of strings
-- into chunks not exceeding @size@ characters. If that is
-- impossible, it uses singleton chunks.
chunksOfSize :: Int -> [String] -> [[String]]
chunksOfSize n = repeatedly $ \xs ->
    let i = length $ takeWhile (<= n) $ scanl1 (+) $ map length xs
    in splitAt (max 1 i) xs
\end{lstlisting}

Writing a small function is easy in Haskell, and even easier to test (we test this function using QuickCheck). Writing it in bash would be infeasible. Reducing the specific behaviours required from external tools leads to significantly less cross-platform concerns.

\subsection{Difficulty of building a DSL}

In any large build system, there are rules (how to build things), and then a very long list of corner cases. For example,

\begin{lstlisting}
compiler_ALEX_OPTS = --latin1
\end{lstlisting}

While this is a concise way of representing it, it's inflexible. Rather than pattern matching on everything, we are pattern matching on two components (which package we are building and which tool we are running). Adding additional filters is impossible.

Implementing the DSL is a nightmare. It's really a configuration setting.


The question of provenance is particularly important in large multi-author build systems, but generally not something that is addressed anywhere - one of our key inovations in \S?. The encapsulation is provided by Haskell modules. The lack of freedom is provided by strong name binding. The problems related to `:' go away by manipulating values rather program text.


\subsection{The consequence of unnecessary complexities}

These unnecessary complexities have a big consequence in terms of complexity and performance. They obscure the underlying real stuff in the build system. Simply removing these already gets us to a much better state.

The main lessons we have learnt are:

\begin{itemize}
\item Abstraction is a powerful and necessary tool. Lack of good abstraction mechanisms is a significant cause of the complexity of previous attempts in \make{}. Functional programming provides good abstractions which in turn allow reducing the complexity at each level, while still obtaining a more powerful result. Marrying build systems and functional programming works well.
\item Use a build system that can express the necessary dependencies. We make use of \textit{monadic dependencies} \S?, \textit{non-file dependencies} \S?, \textit{resources} \S?. While some of these features are only used in a few places (e.g. resources), their absence would require pervasive workarounds, and a significant increase in overall complexity.
\end{itemize}


\section{Abstractions\label{sec:abstractions}}

% \begin{itemize}
%   \item Example shared with Background to Shake section
%   \item top-level, starting with main = buildPackage etc.
% \end{itemize}

In the previous section we covered how Shake helps us sidestep the unnecessary
complexities inherent in a large-scale build system. In this section we focus on
the complexities that remain. We want the build system to be maintainable, correct
and fast, but also:

\begin{itemize}
\item The code should be simple and direct, talking about concepts familiar to
developers, such as files that are built, tools that are used to build
them, and their configuration settings.
\item The code should permit configuration. Many GHC users work with GHC in
different modes and in different environments. Common configuration settings
include turning on/off documentation, choosing different sets of optimisation flags,
etc.
\item The code should permit debugging. There are many obscure configuration
options inherent in any large software project, and the build system is where
such knowledge usually lives. As an example, to include profiling information
when compiling a Haskell source file with GHC we need to pass the \lst'-prof' flag.
We want users to be able to easily determine what flags were used and why.
\end{itemize}
\noindent
To address these challenges we developed a
second DSL for describing configurations on top of Shake.
% A solution was not obvious to us at first, but we ended up building a
% expressions, and an interpreter that evaluates it to run the build.
% Where the DSL would be unnecessarily complex we opt for a direct
% implementation.
The configuration language is tracked (if it changes the affected build rules
are rerun), can have provenance (using the new \emph{implicit locations}
feature\footnote{\scriptsize\urlstyle{sf}\url{https://ghc.haskell.org/trac/ghc/wiki/ExplicitCallStack/ImplicitLocations}$\!\!\!$}
in GHC), and permits easy configuration. As an example:

\begin{lstlisting}
builder Ghc ? way profiling ? arg "-prof"
\end{lstlisting}

\noindent This expression adds the \lst'-prof' argument to the command line when
building a file with GHC in the \lst'profiling' way.

In this section we describe these new abstractions. \S\ref{sec:ghc-context} helps
make our story concrete by focusing on the specifics of GHC's build system,
but otherwise the abstractions we describe are independent of GHC and will, we believe,
be useful to others.

\newcommand{\itab}[1]{\hspace{0em}\rlap{#1}}
\newcommand{\tab}[1]{\hspace{.1\textwidth}\rlap{#1}}
\newcommand{\ctab}[1]{\hspace{.031\textwidth}\rlap{#1}}
\newcommand{\ptab}[1]{\hspace{.074\textwidth}\rlap{#1}}
\newcommand{\cotab}[1]{\hspace{.064\textwidth}\rlap{#1}}
\newcommand{\ttab}[1]{\hspace{.058\textwidth}\rlap{#1}}
\newcommand{\tytab}[1]{\hspace{.06\textwidth}\rlap{#1}}
\newcommand{\atab}[1]{\hspace{.102\textwidth}\rlap{#1}}

\begin{figure}
\begin{lstlisting}
#\vspace{-7mm}#
#\line(1,0){240}#
data PackageType = Library | Program#\hspace{15mm}\textbf{\emph{Build types}}#

data Package = Package
#\itab{~~~~\{} \ctab{pkgName} \ptab{:: PackageName}#
#\itab{~~~~,} \ctab{pkgPath} \ptab{:: FilePath}#
#\itab{~~~~,} \ctab{pkgType} \ptab{:: PackageType \}}#

newtype Way = ... deriving Eq

data Stage = Stage0 | Stage1 | Stage2 | Stage3 deriving#\,#Enum

data Context = Context
#\itab{~~~~\{} \ctab{stage} \cotab{:: Stage}#
#\itab{~~~~,} \ctab{package} \cotab{:: Package}#
#\itab{~~~~,} \ctab{way} \cotab{:: Way \}}#

data Builder = Alex
                  | Ar
                  | GenPrimopCode
                  | Ghc Stage
                  | Haddock
                  ... #\emph{\textrm{plus 22 more builders}}#

data Target = Target
#\itab{~~~~\{} \ctab{context} \ttab{:: Context}#
#\itab{~~~~,} \ctab{builder} \ttab{:: Builder}#
#\itab{~~~~,} \ctab{inputs} \ttab{:: [FilePath]}#
#\itab{~~~~,} \ctab{outputs} \ttab{:: [FilePath] \}}#
#\vspace{-5mm}#
#\line(1,0){240}#
type Expr a = ReaderT Target Action a#\hspace{14mm}\textbf{\emph{Expressions}}#

newtype Diff a = Diff { fromDiff :: a -> a }

type Args = Expr (Diff [String])

append :: [String] -> Args
append as = return $ Diff (<> as)

remove :: [String] -> Args
remove as = return . Diff $ filter (`notElem` as)

arg :: String -> Args
arg = append . return

interpret :: Target -> Args -> Action [String]
interpret target args = do
    diff <- runReaderT args target
    return $ fromDiff diff mempty
#\vspace{-5mm}#
#\line(1,0){240}#
way :: Way -> Expr Bool#\hspace{35mm}\textbf{\emph{Predicates}}#
way w = do
    target <- ask
    return $ way (context target) == w

#\itab{stage} \tytab{:: Stage} \atab{$\rightarrow$ Expr Bool}#
#\itab{package} \tytab{:: Package} \atab{$\rightarrow$ Expr Bool}#
#\itab{builder} \tytab{:: Builder} \atab{$\rightarrow$ Expr Bool}#
#\itab{file} \tytab{:: FilePattern} \atab{$\rightarrow$ Expr Bool}#

(?) :: Monoid a => Expr Bool -> Expr a -> Expr a
predicate ? expr = do
    bool <- predicate
    if bool then expr else return mempty
#\vspace{-4mm}#
\end{lstlisting}
\caption{GHC build system abstractions\label{fig:abstractions}}
\end{figure}

\subsection{GHC-specific abstractions\label{sec:ghc-context}}

We start by describing GHC-specific build types, which form our \emph{build
context}. By abstracting over the context in the subsequent sections we derive a
set of generally useful build abstractions that are applicable to many build
systems.

GHC source code is split into logical units, or \emph{packages}. We model
packages with the \lst'Package' type, see
Figure~\ref{fig:abstractions}~(Build types). A package is identified by a
unique \lst'PackageName' and a \lst'FilePath' pointing to its location in the
source tree. A GHC package can be a library (e.g. \lst'array') or a program
(e.g. \lst'genprimopcode'), which is captured by \lst'PackageType'. There are
32 libraries and 18 programs (the latter includes GHC itself and various utilities).

A package can be built multiple \emph{ways}, for example, to produce a library
with or without profiling information. This is captured by an opaque
type \lst'Way' inhabited by values such as \lst'vanilla' (the simplest
possible way), \lst'profiling' (with profiling information), \lst'debug'
(with debug information), and many others (there are 18 ways in total). Some
ways can be combined, e.g., \lst'debugProfiling'; however, not all combinations
make sense. By making \lst'Way' opaque we make it easier
to add new ways or change their internal representation, something that would be
impossible to achieve in Make, where no information hiding is possible.

In addition to different build ways, each package can be built by several
versions of GHC, which leads to the notion of \emph{stages}.
In \lst'Stage0' we use the \emph{bootstrap} GHC, i.e. the one that is
installed in the system. During this stage we build \lst'Stage1' GHC, an
intermediate compiler that still lacks many features. It is used during the
following \lst'Stage1' for building a fully-featured \lst'Stage2' GHC, the
primary goal of the build system. We sometimes also build \lst'Stage3' GHC as
a self-test: the object code of \lst'Stage2' and \lst'Stage3' compilers
should be the same.

\lst'Stage', \lst'Package' and \lst'Way' form a GHC-specific
\emph{build context} represented by the type \lst'Context'. A typical GHC build
rule, such as \lst'compilePackage', depends on the context as follows: it
uses an appropriate compiler version (e.g. the bootstrap compiler in
\lst'Stage0'), produces object files with different extensions (e.g. vanilla
\lst'*.o' or profiled \lst'*.p_o' object files), puts build artifacts
into an appropriate directory (e.g. \lst'stage1/libraries/base'), etc.

\subsection{General abstractions}

A typical build system invokes several build tools, or \emph{builders}, such as
compilers, linkers, etc., some of which may be built by the
build system itself. The builders are captured by the \lst'Builder' type. It
is useful to distinguish \emph{internal} and \emph{external} builders, i.e.
those that are built by the build system and those which are installed on the
system, respectively. The function \lst'builderProvenance' returns the stage
during which an internal builder is built, the way it is built, and the package
containing the sources (all captured by a \lst'Context'); \lst'Nothing' is
known about the provenance of external builders.

\newcommand{\bptab}[1]{\hspace{.143\textwidth}\rlap{#1}}

\begin{lstlisting}
builderProvenance :: Builder -> Maybe Context
builderProvenance x = case x of
#\itab{~~~GenPrimopCode} \bptab{$\rightarrow$ Just$\,$(Context$\,$Stage0$\,$genprimopcode$\,$vanilla)}#
#\itab{~~~Ghc Stage0} \bptab{$\rightarrow$ Nothing}#
#\itab{~~~Ghc stage} \bptab{$\rightarrow$ Just \$ Context (pred stage) ghc vanilla}#
#\itab{~~~Haddock} \bptab{$\rightarrow$ Just \$ Context Stage2 haddock vanilla}#
#\itab{~~~...}#
#\itab{~~~\_} \bptab{$\rightarrow$ Nothing}#
\end{lstlisting}

\noindent In particular, we can see that \lst'Ghc Stage0' is an external
builder, \lst'Ghc Stage1' is an internal one built from package \lst'ghc'
during \lst'Stage0', \lst'Haddock' is built in \lst'Stage2', etc. There are
27 builders, 16 of which are internal. Furthermore, some builders are
\emph{optional}, e.g., \lst'HsColour', which (if installed) is used to
colourise Haskell code when building documentation.

Each invocation of a builder is fully described by a \lst'Target', which
comprises a build \lst'Context', a \lst'Builder', a list of input files and
a list of output files. 3748 targets are built when building \lst'Stage2' GHC
with documentation (with vanilla and profiled libraries). Consider the following
invocation of \lst'Ghc Stage1' as an example:

\begin{lstlisting}
inplace/bin/ghc-stage1 -O2 -prof -c libraries/base/Prelude.hs
    -o build/stage1/libraries/base/Prelude.p_o
\end{lstlisting}

\noindent The corresponding \lst'Target' is:

\begin{lstlisting}
preludeTarget = Target
#\itab{~~~~\{} \ctab{context} \ttab{= Context Stage1 base profiling}#
#\itab{~~~~,} \ctab{builder} \ttab{= Ghc Stage1}#
#\itab{~~~~,} \ctab{inputs} \ttab{= ["libraries/base/Prelude.hs"]}#
#\itab{~~~~,} \ctab{outputs} \ttab{= ["build/stage1/libraries/base/Prelude.p\_o"] \}}#
\end{lstlisting}

\noindent By examining \lst'preludeTarget' it is possible to compute the full
command line for building \lst'build/stage1/libraries/base/Prelude.p_o':
\begin{itemize}
  \item The builder is \lst'Ghc Stage1'. We lookup the right command
  \lst'inplace/bin/ghc-stage1' with help of \lst'builderProvenance', and use it
  in the following command line template:\vspace{1mm}\\
  \centerline{\!\!\!\!\!\lst'-O2 -c <input> -o <output>'}
  \item The way is \lst'profiling'. We know that we need to add the
  flag \lst'-prof'.$\!\!\!$
  \item We know how to substitute \lst'<input>' and \lst'<output>' in the
  above template.
  \item We ignore \lst'package' since it is not relevant in this particular
  case.
\end{itemize}

\noindent A build system typically contains many such computations (at least one
for each builder) and it is important to provide a terse and readable notation to
describe them. After experimenting with several abstractions, we converged on
\emph{expressions}, as defined in the next subsection.

\subsection{Expressions\label{sec:expressions}}

An expression \lst'Expr a' is a computation that produces a value of type
\lst'Action a' and can read the current build \lst'Target', as shown in
Figure~\ref{fig:abstractions}~(Expressions). For example,
the following expression computes command line arguments for invoking GHC:

\newcommand{\altab}[1]{\hspace{.05\textwidth}\rlap{#1}}

\begin{lstlisting}
ghcArgs :: Expr [String]
ghcArgs = do
    target <- ask
    return $ [ "-O2" ]
        #\hspace{1.5mm}#++ [ "-prof" | way (context target) == profiling ]
        #\hspace{1.5mm}#++ [ "-c", head (inputs target) ]
        #\hspace{1.5mm}#++ [ "-o", head (outputs target) ]
\end{lstlisting}

\subsubsection{Predicates}

The use of conditional \lst'Expr' values is pervasive. In the \lst'ghcArgs' expression
above \lst'-prof' is only applied when profiling. But in fact, the entire expression
is only applicable when using the \lst'Ghc' builder. To make conditionals more concise
we use \emph{predicates} of type \lst'Expr Bool', see
Figure~\ref{fig:abstractions}~(Predicates). In particular, we
use \lst'way :: Way -> Expr Bool' to check which way is currently being built.
For example, predicate \lst'way profiling' returns \lst'True' when the
current target is built using the \lst'profiling' way.

Operator \lst'(?) :: Monoid a => Expr Bool -> Expr a -> Expr a' is a
convenient shortcut for applying a predicate to an expression that computes a monoidal value, such
as \lst'[String]'. For example, the following expression returns
\lst'["-prof"]' when the current target is built the \lst'profiling' way and an
empty list of arguments otherwise:

\begin{lstlisting}
prof :: Expr [String]
prof = way profiling ? return ["-prof"]
\end{lstlisting}

\noindent Expressions computing monoids themselves form a monoid:

\begin{lstlisting}
instance Monoid a => Monoid (Expr a) where
    mempty #\hspace{1.5mm}#= return mempty
    mappend = liftM2 mappend
\end{lstlisting}

\noindent Predicates and monoidal expressions are a powerful combination with
many useful laws that allow us to reason about them:
\begin{enumerate}
  \item \itab{Absorption:} \tab{\lst'p ? mempty === mempty'}
  \item \itab{Distributivity:} \tab{\lst'p ? (e <> f) === p ? e <> p ? f'}
  \item \itab{Conjunction:} \tab{\lst'p ? q ? e === (p &&& q) ? e === q ? p ? e'}
  \item \itab{Disjunction:} \tab{if \lst'p &&& q === False' then} \vspace{1mm}\\
  \lst'    p ? e <> q ? e === (p ||| q) ? e === q ? e <> p ? e'
  \item \itab{Complement:} \tab{\lst'p ? e <> !!!p ? e === e'}
\end{enumerate}

\subsubsection{Extensibility}

All expressions need to be modifiable by users of the build system. We therefore
need to provide a way not only to add new arguments, but also
to modify and remove them. A simple solution is to switch to difference
list expressions, represented by the type \lst'Expr (Diff a)', which is used to
construct values of type \lst'Diff a' with the following monoid instance:

\begin{lstlisting}
instance Monoid (Diff a) where
    mempty #\hspace{24mm}#= Diff id
    mappend (Diff x) (Diff y) = Diff $ y . x
\end{lstlisting}

\noindent The reverse order of function composition \lst'y . x' ensures that
when two \lst'Expr (Diff a)' computations are combined \lst'c1 <> c2', then
\lst'c1' is applied first and \lst'c2' is applied second.

The following functions can be used to append and remove items to/from a
difference list:

\begin{lstlisting}
append :: Monoid a => a -> Expr (Diff a)
append x = return $ Diff (<> x)

remove :: Eq a => [a] -> Expr (Diff [a])
remove xs = return . Diff $ filter (`notElem` xs)
\end{lstlisting}


We are now ready to introduce \lst'Args', a type of expression for
constructing command line arguments in the build system. In addition to the
above generic functions (whose specialised versions are shown in
Figure~\ref{fig:abstractions}), it is equipped with the function
\lst'arg :: String -> Args' for injecting simple \lst'String' arguments into
an expression, e.g. \lst'arg "-prof" :: Args'.

With these abstractions in place, we can construct command
line arguments for GHC as follows:

\begin{lstlisting}
ghcArgs :: Args
ghcArgs = builder Ghc ? mconcat
    [ arg "-O2"
    , way profiling ? arg "-prof"
    , arg "-c", arg =<< getInput
    , arg "-o", arg =<< getOutput ]
\end{lstlisting}

\noindent Here \lst'getInput :: Expr FilePath' and
\lst'getOutput :: Expr FilePath' are expressions that check that
\lst'Target.inputs' and \lst'Target.outputs' contain exactly one element and
return it.

The resulting \lst'ghcArgs' expression is terse and readable. All
distracting plumbing details have been abstracted away so that the designers and
users of the build system can focus on what matters.

%\todo{\textbf{Andrey}: Also mention Packages and Ways expressions.}

We compose all command line arguments into a single \lst'args' expression,
applying custom user modifications \lst'userArgs' to the end,
allowing the user to override any default setting:

\begin{lstlisting}
args :: Args
args = mconcat [ ghcArgs, ..., userArgs ]
\end{lstlisting}

The resulting expression is used in the \lst'build' function that is responsible
for building a given \lst'Target':

\begin{lstlisting}
build :: Target -> Action ()
build target@Target {..} = do
    path <- builderPath builder
    need [path]
    checkArgsHash target
    argList <- interpret target args
    cmd [path] argList
\end{lstlisting}

\noindent The \lst'build' function proceeds as follows:
\begin{itemize}
  \item First, \lst'builderPath :: Builder -> Action FilePath' determines the
  path to the builder depending on its provenance and the contents of the
  \lst'system.config' file.
  \item We \lst'need' the builder to make sure it is up-to-date. Some builders
  are built by the build system, e.g. \lst'genprimopcode',
  \lst'ghc-cabal', \lst'Stage1' GHC, so it is important to rebuild them (as well
  as all dependent targets) if needed.
  \item The function \lst'checkArgsHash :: Target -> Action ()' checks whether the
  \lst'target' needs to be rebuilt because the command line computed from
  \lst'args' expression has changed since the previous build. If it has,
  the target is rebuilt even if it is otherwise up-to-date. This step tracks changes both in
  the environment and in the build system itself. We
  track command line hashes using polymorphic dependencies, see~\S\ref{sec:polymorphic}.
  \item We \lst'interpret' the \lst'args' expression w.r.t. to the
  \lst'target', and obtain the list \lst'argList :: [String]' of arguments to be
  passed to the builder. See Figure~\ref{fig:abstractions} for the implementation of
  \lst'interpret'.
  \item Finally, we invoke the builder with appropriate arguments using Shake's
  \lst'cmd' function.
\end{itemize}

\subsection{A simple build system example\label{sec:build-example}}

Figure~\ref{fig:example-abstractions} shows a simple build system that uses the
abstractions introduced in this section.

\newcommand{\tabx}[1]{\hspace{.106\textwidth}\rlap{#1}}
\newcommand{\taby}[1]{\hspace{.074\textwidth}\rlap{#1}}
\newcommand{\tabz}[1]{\hspace{.24\textwidth}\rlap{#1}}

\begin{figure}
\begin{lstlisting}
#\vspace{-7mm}#
#\line(1,0){240}#
#\itab{data Version} \tabx{= Head} \taby{| Release}\hspace{34.5mm}\textbf{\emph{Build~types}}#
#\itab{data Package} \tabx{= Array} \taby{| Base}#
#\itab{data Way} \tabx{= Vanilla} \taby{| Profiling}#

#\itab{data Context} \tabx{= Context Version Package Way}#

#\itab{data Builder} \tabx{= Ghc Version | Ar}#
#\vspace{-5mm}#
#\line(1,0){240}#
args :: Args#\hspace{33mm}\textbf{\emph{Command~line~arguments}}#
args = mconcat
    [ builder Ghc ? mconcat
        [ arg "-O2"
        , way Profiling ? arg "-prof"
        , arg "-c", arg =<< getInput
        , arg "-o", arg =<< getOutput ]

    , builder Ar ? mconcat
        [ arg "q"
        , arg =<< getOutput
        , append =<< getInputs ]

    , file "//base/GHC/IO.*" ? arg "-funbox-strict-fields"

    , builder Ghc ? package Array ? arg "-Wall" ]
#\vspace{-5mm}#
#\line(1,0){240}#
buildPackage :: Context -> Rules ()#\hspace{20.5mm}\textbf{\emph{Build rules}}#
buildPackage context@Context {..} = do
    path context </> "*" ++ osuf way %> \obj -> do
        let src = obj -<.> "hs"
        deps <- lookupDependencies context obj
        need $ src : deps
        build $ Target context (Ghc version) [src] [obj]

    path context </> "*" ++ asuf way %> \a -> do
        srcs <- lookupSources context
        let objs = [ src -<.> osuf way | src <- srcs ]
        need objs
        build $ Target context Ar objs [a]

path :: Context -> FilePath
path Context#\,#{..} = show#\,#version </> pkgName#\,#package

osuf :: Way -> String
osuf Vanilla   = ".o"
osuf Profiling = ".p_o"

asuf :: Way -> String
asuf Vanilla   = ".a"
asuf Profiling = "_p.a"

lookupDependencies :: Context#\,#->#\,#FilePath#\,#->#\,#Action#\,#[FilePath]
lookupDependencies context src = do ...

lookupSources :: Context -> Action [FilePath]
lookupSources context = do ...
#\vspace{-5mm}#
#\line(1,0){240}#
main :: IO ()#\hspace{59.5mm}\textbf{\emph{Main}}#
main = shake shakeOptions $ do
    for_ [Head, Release] $ \version ->
        for_ [Array, Base] $ \package ->
            for_ [Vanilla, Profiling] $ \way -> do
                let context = Context version package way
                want [#\,#path#\,#context </> "HSlib" ++ asuf#\,#way#\,#]
                buildPackage context
#\vspace{-4mm}#
\end{lstlisting}

\caption{Example of a build system\label{fig:example-abstractions}}
\end{figure}

The build system comprises two packages (\lst'Array' and \lst'Base') that can
be built in two ways (\lst'Vanilla' and \lst'Profiling'), using two versions
of GHC (\lst'Ghc Head' and \lst'Ghc Release') and an archiver tool \lst'Ar'.
Note, the build \lst'Context' is slightly different from GHC, but this does not
prevent us from using the same abstractions.

\newcommand{\checkedbox}{\makebox[0pt][l]{$\square$}\raisebox{.15ex}{\hspace{0.1em}$\checkmark$}}
\newcommand{\uncheckedbox}{\makebox[0pt][l]{$\square$}\raisebox{.15ex}{\hspace{0.92em}}}
\begin{table*}[t]
\centering
\begin{tabular}{p{60mm} || p{50mm} | p{50mm}}
\textbf{Use case}
& \textbf{Old build system} based on Make
& \textbf{New build system} based on Shake
\\
\hline
\textsf{U1:} Fully-featured GHC build
& Everything is built \hfill \checkedbox
& Not all features supported, see~\S\ref{sec:limitations} \hfill \uncheckedbox
\\
\textsf{U2:} Clean build
& Everything is built \hfill \checkedbox
& Everything is built \hfill \checkedbox
\\
\textsf{U3:} Zero build \hspace{6.4mm}
& Nothing is rebuilt \hfill \checkedbox
& Nothing is rebuilt \hfill \checkedbox
\\
\hline
\textsf{U4:} Touch: \hspace{10.2mm}\textsf{libraries/base/Prelude.hs}
& \textsf{Prelude.o}, \textsf{base} library, and all \hfill \uncheckedbox
\newline dependent binaries are rebuilt
& Nothing is rebuilt \hfill \checkedbox
\\
\textsf{U5:} Add comment: \textsf{libraries/base/Prelude.hs}
& \textsf{Prelude.o}, \textsf{base} library, and all \hfill \uncheckedbox
\newline dependent binaries are rebuilt
& Only \textsf{Prelude.o} is rebuilt \hfill \checkedbox
\\
\textsf{U6:} Modify code: \hspace{1.75mm}\textsf{libraries/base/Prelude.hs}
& \textsf{Prelude.o} and all its dependencies \hfill \checkedbox \newline
are rebuilt
& \textsf{Prelude.o} and all its dependencies \hfill \checkedbox \newline
are rebuilt
\\
\textsf{U7:} Add comment: \textsf{utils/ghc-cabal/Main.hs}
& Almost everything is rebuilt \hfill \uncheckedbox
& All \textsf{ghc-cabal} rules are rerun \hfill \uncheckedbox
\\
\textsf{U8:} Modify code: \hspace{1.75mm}\textsf{utils/ghc-cabal/Main.hs}
& Almost everything is rebuilt \hfill \uncheckedbox
& Only the targets affected by the \hfill \checkedbox \newline change are rebuilt
\\
\hline
\textsf{U9:} $\textit{~~}$Modify the build system: pass \textsf{-O2} when
\newline $\textit{~~~~~~~~~}$compiling Stage2 GHC
& Nothing is rebuilt \hfill \uncheckedbox
& Stage2 GHC and its dependencies \hfill \checkedbox \newline are rebuilt
\\
\textsf{U10:} Modify the build system without changing \newline
$\textit{~~~~~~~~~}$command line arguments of build tools
& Nothing is rebuilt \hfill \uncheckedbox
& Nothing is rebuilt \hfill \uncheckedbox
\\
% \hline
% \textsf{U10:} Switch to a \textsf{git} branch and back
% & Some files are rebuilt \hfill \uncheckedbox
% & Nothing is rebuilt \hfill \checkedbox
% \\
\textsf{U11:} Change path to \textsf{gcc}
& Everything is rebuilt \hfill \uncheckedbox
& Files depending on \textsf{gcc} are rebuilt \hfill \checkedbox
\\
\end{tabular}
\caption{Comparison of GHC build systems on common use cases. Checkmarks
\checkmark indicate desired behaviour.}
\label{tab:use-cases}
\end{table*}

Build rules are generated by the \lst'buildPackage' function, which given a
build \lst'Context', describes how to compile a Haskell source file and build a
library by archiving the obtained object files. \lst'buildPackage' relies on
several helper functions, which define the \lst'path' to build artifacts,
set way-specific extensions for object and archive files, lookup
dependencies of a source file, and compute a list of source files for a given
library. Command line arguments are specified as a single expression~\lst'args',
which makes use of package-, way-, builder-, and file-specific arguments.

The \lst'main' function is straightforward: for each possible \lst'Context'
we request the corresponding library to be built using \lst'want', as well as
generate necessary rules by calling \lst'buildPackage'. If we run the build
system it will build 8 libraries and all associated object files.


\newcommand{\checkedbox}{\makebox[0pt][l]{$\square$}\raisebox{.15ex}{\hspace{0.1em}$\checkmark$}}
\newcommand{\uncheckedbox}{\makebox[0pt][l]{$\square$}\raisebox{.15ex}{\hspace{0.92em}}}
\begin{table*}[t]
\centering
\begin{tabular}{p{60mm} || p{50mm} | p{50mm}}
\textbf{Use case}
& \textbf{Old build system} based on Make
& \textbf{New build system} based on Shake
\\
\hline
\textsf{U1:} Fully-featured GHC build
& Everything is built \hfill \checkedbox
& Not all features supported, see~\S\ref{sec:limitations} \hfill \uncheckedbox
\\
\textsf{U2:} Clean build
& Everything is built \hfill \checkedbox
& Everything is built \hfill \checkedbox
\\
\textsf{U3:} Zero build \hspace{6.4mm}
& Nothing is rebuilt \hfill \checkedbox
& Nothing is rebuilt \hfill \checkedbox
\\
\hline
\textsf{U4:} Touch: \hspace{10.2mm}\textsf{libraries/base/Prelude.hs}
& \textsf{Prelude.o}, \textsf{base} library, and all \hfill \uncheckedbox
\newline dependent binaries are rebuilt
& Nothing is rebuilt \hfill \checkedbox
\\
\textsf{U5:} Add comment: \textsf{libraries/base/Prelude.hs}
& \textsf{Prelude.o}, \textsf{base} library, and all \hfill \uncheckedbox
\newline dependent binaries are rebuilt
& Only \textsf{Prelude.o} is rebuilt \hfill \checkedbox
\\
\textsf{U6:} Modify code: \hspace{1.75mm}\textsf{libraries/base/Prelude.hs}
& \textsf{Prelude.o} and all its dependencies \hfill \checkedbox \newline
are rebuilt
& \textsf{Prelude.o} and all its dependencies \hfill \checkedbox \newline
are rebuilt
\\
\textsf{U7:} Add comment: \textsf{utils/ghc-cabal/Main.hs}
& Almost everything is rebuilt \hfill \uncheckedbox
& All \textsf{ghc-cabal} rules are rerun \hfill \uncheckedbox
\\
\textsf{U8:} Modify code: \hspace{1.75mm}\textsf{utils/ghc-cabal/Main.hs}
& Almost everything is rebuilt \hfill \uncheckedbox
& Only the targets affected by the \hfill \checkedbox \newline change are rebuilt
\\
\hline
\textsf{U9:} Modify the build system: pass \textsf{-O2} when \newline
$\textit{~~~~~~~}$compiling Stage2 GHC
& Nothing is rebuilt \hfill \uncheckedbox
& Stage2 GHC and its dependencies \hfill \checkedbox \newline are rebuilt
\\
\textsf{U10:} Modify the build system without changing \newline
$\textit{~~~~~~~}$command line arguments of build tools
& Nothing is rebuilt \hfill \uncheckedbox
& Nothing is rebuilt \hfill \uncheckedbox
\\
% \hline
% \textsf{U10:} Switch to a \textsf{git} branch and back
% & Some files are rebuilt \hfill \uncheckedbox
% & Nothing is rebuilt \hfill \checkedbox
% \\
\textsf{U11:} Change path to \textsf{gcc}
& Everything is rebuilt \hfill \uncheckedbox
& Files depending on \textsf{gcc} are rebuilt \hfill \checkedbox
\\
\end{tabular}
\caption{Comparison of GHC build systems on common use cases. Checkmarks
\checkmark indicate desired behaviour.}
\label{tab:use-cases}
\end{table*}

\section{Shaking up GHC\label{sec:ghc}}

In this section we report on our experience of applying the techniques presented
so far to building a large-scale software project: the Glasgow Haskell Compiler.

\subsection{Current status and limitations\label{sec:limitations}}

We implemented a new build system for GHC from scratch using Shake and our build
abstractions from~\S\ref{sec:abstractions}. The current implementation can build
\lst'Stage2' GHC, but has the following limitations:
\begin{itemize}
  \item We only build \lst'vanilla' and \lst'profiling' way.
  \item We reuse GHC testing infrastructure of the old build system.
  \item Only HTML documentation can be built.
  \item Not all build flavours and command line flags are supported.
  \item Cross-compilation is not implemented.
  \item Installation or binary/source distribution are not supported.
\end{itemize}

\noindent We intend to fix the above in the near future; nothing presents any
new challenges or requires changes to the build infrastructure.

\subsection{Qualitative analysis\label{sec:use-cases}}

In this section we discuss several use cases of the GHC build system, which
are fairly typical for build systems in general. Table~\ref{tab:use-cases}
lists use cases \lst'U1-U11' highlighting differences between the old and the
new build systems. Below we go through some of the use cases in more detail.
See~\S\ref{sec:benchmarks} for performance comparison.

\lst'U1-U3' are simplest use cases. For the sake of fairness we start with
\lst'U1', where the old build system reigns over our current implementation due
to the aforementioned limitations. When unsupported features are not used
(\lst'U2'), the new build system successfully builds all expected targets.
Running a build system twice in a row must be equivalent to only running it
once; the second build must do nothing, hence the name \emph{zero
build} (\lst'U3'). The new build system works as expected, and is much faster
than the old one (see \S\ref{sec:benchmarks}).

% The time it takes represents internal overheads of the build system:
% scanning the file system, reading the database, etc.

In \lst'U4-U6' we modify \lst'libraries/base/Prelude.hs' and rebuild GHC.
If we \lst'touch' the file (\lst'U4'), i.e. change only its modification
time, the new build system rebuilds nothing, as desired. The old build system rebuilds
\lst'Prelude.o', the \lst'base' library, and all dependent binaries, such as
\lst'Stage2' GHC. This use case commonly occurs when switching \lst'git'
branches, as explained in~\S\ref{sec:file-contents}, or whenever a user changes
a file, but then decides to undo the changes. In \lst'U5' we add comments,
forcing the new build system to recompile \lst'Prelude.hs'. It then notices the
object code is unchanged, and stops: there is nothing else to be done. The old
build system continues to rebuild the \lst'base' library and dependent binaries,
which is unnecessary. In \lst'U6' the modification of \lst'Prelude.hs' leads to
changes in \lst'Prelude.o', which causes all dependencies to be rebuilt. Both
build systems handle this case correctly.

\lst'U7-U8' are similar, but we now modify sources of \lst'ghc-cabal' build
tool. The old build system rebuilds almost everything in both cases,
which is unnecessary. Rebuilds are caused by rerunning updated \lst'ghc-cabal'
binary, which changes modification time of generated \lst'package-data.mk'
files. The lack of polymorphic dependencies means we have to depend on the whole
file when using Make, therefore even if only one field in a
\lst'package-data.mk' file is changed, e.g. \lst'CC_OPTS', we end up
rebuilding everything, not only C compilation rules that depend on
\lst'CC_OPTS'. The new build system uses Shake's polymorphic
dependencies~\S\ref{sec:polymorphic} to avoid such unnecessary rebuilds. Note
however, that \lst'U7' behaviour is still suboptimal: \lst'ghc-cabal' rules
are rerun, because GHC currently produces non-deterministic output
(\lst'ghc-cabal''s binary is changed).

In \lst'U9' we modify the build system itself by changing command line
arguments for one of the build files. The old build system rebuilds nothing,
as Make does not track such changes. The new build system correctly reruns all
affected rules. We currently only track command line arguments, therefore other,
more subtle modifications of the build system (\lst'U10') are unnoticed.

% Working with multiple \lst'git' branches is a problematic use-case for build
% systems based on Make (as discissed in the second bullet point of
% \S\ref{sec:file-contents}). In \lst'U10' we conduct the following simple
% experiment:
% 
% \begin{lstlisting}
% $ build
% $ git checkout -b test
% $ echo -- comment >> libraries/base/Prelude.hs
% $ git checkout master
% $ build
% \end{lstlisting}
% 
% We expect the second \lst'build' to do nothing, since the \lst'master'
% branch has not changed. However, \lst'git checkout master' changes modification
% time of all files that were updated between branches, forcing the new
% build system to rerun all dependent rules (as in \lst'U4'). Shake's ability to
% track file contents in addition to modification time~\S\ref{sec:file-contents}
% allows us to avoid such unnecessary rebuils.

In \lst'U11' we modify the build environment, by changing the path to \lst'gcc'
in the configuration files. As in \lst'U7-U8', the old build system rebuilds
almost everything since depending on a single configuration setting is not
supported. The new build system correctly reruns only affected rules.
\todo{\textbf{Andrey}: Add \lst'U12', edit cabal.in case.}

In summary, the new build system correctly handles most use cases, whereas
the old one performs a lot of unnecessary rebuilds in all but simple cases.

\subsection{Benchmarks\label{sec:benchmarks}}

When building from scratch, ignoring the initial \lst'boot'/\lst'configure'
steps which are shared, the old build system takes 1266 seconds on Windows and 
649 seconds on Linux; while the new build system takes 737 seconds on Windows and
578 seconds on Linux. These were tested in as similar configurations as we could
manage (by disabling all the features the new system does not support),
but due to the complexity of the build systems, there are almost certainly
minor differences. For the zero build, the old build system takes 12.3
seconds on Windows and 2.2 seconds on Linux; while the new one takes
2.1 seconds on Windows and 2.0 seconds on Linux. All measurements were collected
using \lst'-j4'.

Since Shake and Haskell both provide profiling and analysis tools, we have
already used these features to optimise the new build system, resulting in
modest gains so far and several opportunities we have yet to exploit. Looking at
the current full build time of 737 seconds, the longest single task is
building the GMP library (315 seconds), and the total time of single-threaded
computation is 2206 seconds, of which 2099 is calling out to external processes.
The critical dependency chain has 378 steps in it, and requires 463 seconds -- a
lower bound on the time even with an unlimited number of processors.

\section{Related Work\label{section-review}}

This paper is about writing build systems at scale, a subject without much
literature since \citet{miller:recursive_make}. When
\citet{mcintosh:build_maintenance_effort} studied software maintenance they
found that build systems can take up to 27\% of
development effort, and that improvements to the build system rapidly paid off.
Recently \citet{martin:make_it_simple} surveyed which Make features are used,
and then \citet{martin:maintenance_complexity_makefiles} classified them by
complexity -- unsurprisingly they found that as Makefiles grow, their complexity
increases, and that the features required for hand written Makefiles are those
which are most complex. In the remainder of this section we focus on some of the
features found in other build systems which could be useful at scale.

\subsection{Embedded language}

A build system can either be specified using structured metadata, e.g. Bazel
\cite{bazel}, or embedded into a standard programming language -- for example
SCons in Python \cite{scons}, Pluto in Java \cite{pluto} and Jenga in OCaml
\cite{jenga}. For complex bespoke build systems, embedding into a language
allows both complex operations (\S\ref{sec:real_code}) and better abstractions
(\S\ref{sec:abstractions}) -- essentially allowing us to write most of our build
system in a domain language tailored to our specific project.

Even sticking to Haskell as the embedded language, there are a surprisingly
large number of libraries implementing a dependency aware build system -- we
know of eleven in addition to Shake (Abba, Blueprint, Buildsome, Coadjute, Cake
$\times$ 2, Hake, Hmk, Nemesis, OpenShake and Zoom). Of these, the two Cake
libraries and OpenShake are based on an early presentation of the principles behind Shake.

\subsection{Advanced dependencies}

We have found that while powerful dependencies might only be used in a few
places, if they are missing the workarounds can be pervasive
(\S\ref{sec:dynamic-deps}). A few build systems support resources, e.g.
Ninja \cite{ninja}, and several support monadic dependencies (e.g. Redo
\cite{redo}, Jenga, Pluto, SCons). A few build systems directly support
dependency features more powerful than Shake, for example Pluto supports rules
that run until a fixed-point is reached and rules whose output is not known in
advance. These features can be encoded in Shake, but are not present natively.

\subsection{Automatic dependency management}

In both Shake and Make, all dependencies must be declared explicitly. However,
in build systems such as Tup \cite{tup} and Buildsome \cite{buildsome}, some
dependencies are automatically captured by monitoring program execution, albeit
only \emph{after} the dependency has been used (like \lst'needed' in
\S\ref{sec:needed}). The Fabricate tool \cite{fabricate} takes a unique approach
to defining build systems, providing a series of steps that run sequentially,
but are skipped if their automatically-detected inputs have not changed.
Unfortunately no cross-platform APIs are available to detect used dependencies,
so such tools are all limited in which platforms they support.

\subsection{Build clusters}

The build systems Bazel and Buck \cite{buck} are used at Google and Facebook
respectively, both operating at sizes significantly beyond that of the GHC build
system (reportedly billions of lines of code). Both systems take a metadata
approach, with various rule types baked in. As an example, the \lst'cxx_binary'
rule builds a C/C++ binary given a list of source files and dependencies,
taking care of suitable build flags and conventions, much like
\lst'buildPackage' from~\S\ref{sec:build-example} but a lot more feature-rich.
The disadvantage of such an approach is that the available rules are fixed,
making it difficult to encode something like a bootstrapping compiler.
Generating source code is not really supported -- a problem typically solved by
committing generated files to version control. Both tools also support build
clusters, which build code once and share the resulting objects to everyone
without recompiling locally -- an essential feature at such scales.

\section{Conclusions and future work\label{section-conclusions}}

We have demonstrated that Make really is unsuitable for large complex build
systems, regardless of whether used recursively or non-recursively. Using Shake
we have rewritten the GHC build system, producing the fifth and hopefully final
version. While all previous versions have started simple and gained complexity
as they progressed, this version is different. Developing the abstractions in
\S\ref{sec:abstractions} took many months of discussion and refinement. Once the
fundamental concepts were in place, the rest was ``just'' coding and reverse
engineering the existing system. The result is faster, more maintainable and
more correct. There are three major tasks remaining for future work:

\begin{itemize}
\item While we have demonstrated that our approach works, we have not yet
implemented all features of the build system, and hope to do so over the next
few months. Once complete, we expect it to quickly become the only supported
method of building GHC.

\item Our abstractions from \S\ref{sec:abstractions} were designed to allow
tracking provenance of command line arguments -- mapping each flag to the
location of the expression that generated it. This feature relies on the
\emph{implicit locations} feature of the latest GHC.

% using the new \emph{implicit locations}
% feature\footnote{\scriptsize\urlstyle{sf}\url{https://ghc.haskell.org/trac/ghc/wiki/ExplicitCallStack/ImplicitLocations}$\!\!\!$}
% in GHC

\item While faster than the old system, the build is still slower than we would
like. The zero build time could be reduced by switching to a more performant serialisation
library. The critical path of a full rebuild takes over seven minutes, limiting
the gains available from additional processors. We hope to break this critical path
by refactoring -- a feasible task in the new system.
\end{itemize}


% \acks
% Thanks to Standard Chartered, where Shake was initially developed.

% \vspace{2mm}
% \noindent
% \footnotesize{Neil Mitchell is employed by Standard Chartered Bank. This paper has been created in a personal capacity and Standard Chartered Bank does not accept liability for its content. Views expressed in this paper do not necessarily represent the views of Standard Chartered Bank.}

\bibliographystyle{plainnat}
\balance
\bibliography

\end{document}
