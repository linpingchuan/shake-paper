\section{Abstractions\label{sec:abstractions}}

% \begin{itemize}
%   \item Example shared with Background to Shake section
%   \item top-level, starting with main = buildPackage etc.
% \end{itemize}

In the previous section we covered how Shake helps us sidestep the unnecessary
complexities inherent in a large-scale build system. In this section we focus on
the complexities that remain. We want the build system to be maintainable, correct
and fast, but also:

\begin{itemize}
\item The code should be simple and direct, talking about concepts familiar to
developers, such as files that are built, tools that are used to build
them, and their configuration settings.
\item The code should permit configuration. Many GHC users work with GHC in
different modes and in different environments. Common configuration settings
include turning on/off documentation, choosing different sets of optimisation flags,
etc.
\item The code should permit debugging. There are many obscure configuration
options inherent in any large software project, and the build system is where
such knowledge usually lives. As an example, to include profiling information
when compiling a Haskell source file with GHC we need to pass the \lst'-prof' flag.
We want users to be able to easily determine what flags were used and why.
\end{itemize}
\noindent
To address these challenges we developed a
second DSL for describing configurations on top of Shake.
% A solution was not obvious to us at first, but we ended up building a 
% expressions, and an interpreter that evaluates it to run the build.
% Where the DSL would be unnecessarily complex we opt for a direct
% implementation.
The configuration language is tracked (if it changes the affected build rules
are rerun), can have provenance (using the new \emph{implicit locations}
feature\footnote{\scriptsize\urlstyle{sf}\url{https://ghc.haskell.org/trac/ghc/wiki/ExplicitCallStack/ImplicitLocations}$\!\!\!$}
in GHC), and permits easy configuration. As an example:

\begin{lstlisting}
builderGhc ? way profiling ? arg "-prof"
\end{lstlisting}

\noindent This expression adds the \lst'-prof' argument to the command line when
building a file with GHC in the \lst'profiling' way.

In this section we describe these new abstractions. \S\ref{sec:ghc-context} helps
make our story concrete by focusing on the specifics of GHC's build system,
but otherwise the abstactions we describe are independent of GHC and will, we believe,
be useful to others.

\newcommand{\itab}[1]{\hspace{0em}\rlap{#1}}
\newcommand{\tab}[1]{\hspace{.1\textwidth}\rlap{#1}}
\newcommand{\ctab}[1]{\hspace{.031\textwidth}\rlap{#1}}
\newcommand{\ptab}[1]{\hspace{.074\textwidth}\rlap{#1}}
\newcommand{\cotab}[1]{\hspace{.064\textwidth}\rlap{#1}}
\newcommand{\ttab}[1]{\hspace{.058\textwidth}\rlap{#1}}
\newcommand{\tytab}[1]{\hspace{.06\textwidth}\rlap{#1}}
\newcommand{\atab}[1]{\hspace{.102\textwidth}\rlap{#1}}

\begin{figure}
\begin{lstlisting}
#\vspace{-7mm}#
#\line(1,0){240}#
data PackageType = Library | Program#\hspace{15mm}\textbf{\emph{Build types}}#

data Package = Package
#\itab{~~~~\{} \ctab{pkgName} \ptab{:: PackageName}#
#\itab{~~~~,} \ctab{pkgPath} \ptab{:: FilePath}#
#\itab{~~~~,} \ctab{pkgType} \ptab{:: PackageType \}}#

newtype Way = ... deriving Eq

data Stage = Stage0 | Stage1 | Stage2 | Stage3 deriving#\,#Enum

data Context = Context
#\itab{~~~~\{} \ctab{stage} \cotab{:: Stage}#
#\itab{~~~~,} \ctab{package} \cotab{:: Package}#
#\itab{~~~~,} \ctab{way} \cotab{:: Way \}}#

data Builder = Alex
                  | Ar
                  | GenPrimopCode
                  | Ghc Stage
                  | Haddock
                  ... #\emph{\textrm{plus 22 more builders}}#

data Target = Target
#\itab{~~~~\{} \ctab{context} \ttab{:: Context}#
#\itab{~~~~,} \ctab{builder} \ttab{:: Builder}#
#\itab{~~~~,} \ctab{inputs} \ttab{:: [FilePath]}#
#\itab{~~~~,} \ctab{outputs} \ttab{:: [FilePath] \}}#
#\vspace{-5mm}#
#\line(1,0){240}#
type Expr a = ReaderT Target Action a#\hspace{14mm}\textbf{\emph{Expressions}}#

newtype Diff a = Diff { fromDiff :: a -> a }

type Args = Expr (Diff [String])

append :: [String] -> Args
append as = return $ Diff (<> as)

remove :: [String] -> Args
remove as = return . Diff $ filter (`notElem` as)

arg :: String -> Args
arg = append . return

interpret :: Target -> Args -> Action [String]
interpret target args = do
    diff <- runReaderT args target
    return $ fromDiff diff mempty
#\vspace{-5mm}#
#\line(1,0){240}#
way :: Way -> Expr Bool#\hspace{35mm}\textbf{\emph{Predicates}}#
way w = do
    target <- ask
    return $ way (context target) == w

#\itab{stage} \tytab{:: Stage} \atab{$\rightarrow$ Expr Bool}#
#\itab{package} \tytab{:: Package} \atab{$\rightarrow$ Expr Bool}#
#\itab{builder} \tytab{:: Builder} \atab{$\rightarrow$ Expr Bool}#
#\itab{file} \tytab{:: FilePattern} \atab{$\rightarrow$ Expr Bool}#

(?) :: Monoid a => Expr Bool -> Expr a -> Expr a
predicate ? expr = do
    bool <- predicate
    if bool then expr else return mempty
#\vspace{-4mm}#
\end{lstlisting}
\caption{GHC build system abstractions\label{fig:abstractions}}
\end{figure}

\subsection{GHC-specific abstractions\label{sec:ghc-context}}

We start by describing GHC-specific build types, which form our \emph{build
context}. By abstracting over the context in the subsequent sections we derive a
set of generally useful build abstractions that are applicable to many build
systems.

GHC source code is split into logical units, or \emph{packages}. We model
packages with the \lst'Package' type, see
Figure~\ref{fig:abstractions}~(Build types). A package is identified by a
unique \lst'PackageName' and a \lst'FilePath' pointing to its location in the
source tree. A GHC package can be a library (e.g. \lst'array') or a program
(e.g. \lst'genprimopcode'), which is captured by \lst'PackageType'. There are
32 libraries and 18 programs (the latter includes GHC itself and various utilities).

A package can be built multiple \emph{ways}, for example, to produce a library
with or without profiling information. This is captured by an opaque
type \lst'Way' inhabited by values such as \lst'vanilla' (the simplest
possible way), \lst'profiling' (with profiling information), \lst'debug'
(with debug information), and many others (there are 18 ways in total). Some
ways can be combined, e.g., \lst'debugProfiling'; however, not all combinations
make sense. By making \lst'Way' opaque we make it easier
to add new ways or change their internal representation, something that would be
impossible to achieve in Make, where no information hiding is possible.

In addition to different build ways, each package can be built by several
versions of GHC, which leads to the notion of \emph{stages}.
In \lst'Stage0' we use the \emph{bootstrap} GHC, i.e. the one that is
installed in the system. During this stage we build \lst'Stage1' GHC, an
intermediate compiler that still lacks many features. It is used during the
following \lst'Stage1' for building a fully-featured \lst'Stage2' GHC, the
primary goal of the build system. We sometimes also build \lst'Stage3' GHC as
a self-test: the object code of \lst'Stage2' and \lst'Stage3' compilers
should be the same.

\lst'Stage', \lst'Package' and \lst'Way' form a GHC-specific
\emph{build context} represented by the type \lst'Context'. A typical GHC build
rule, such as \lst'compilePackage', depends on the context as follows: it
uses an appropriate compiler version (e.g. the bootstrap compiler in
\lst'Stage0'), produces object files with different extensions (e.g. vanilla
\lst'*.o' or profiled \lst'*.p_o' object files), puts build artefacts
into an appropriate directory (e.g. \lst'stage1/libraries/base'), etc.

\subsection{General abstractions}

A typical build system invokes several build tools, or \emph{builders}, such as
compilers, linkers, etc., some of which may be built by the
build system itself. The builders are captured by the \lst'Builder' type. It
is useful to distinguish \emph{internal} and \emph{external} builders, i.e.
those that are built by the build system and those which are installed on the
system, respectively. The function \lst'builderProvenance' returns the stage
during which an internal builder is built, the way it is built, and the package
containing the sources (all captured by a \lst'Context'); \lst'Nothing' is
known about the provenance of external builders.

\newcommand{\bptab}[1]{\hspace{.143\textwidth}\rlap{#1}}

\begin{lstlisting}
builderProvenance :: Builder -> Maybe Context
builderProvenance x = case x of
#\itab{~~~Ghc Stage0} \bptab{$\rightarrow$ Nothing}#
#\itab{~~~Ghc stage} \bptab{$\rightarrow$ Just \$ Context (pred stage) ghc vanilla}#
#\itab{~~~Haddock} \bptab{$\rightarrow$ Just \$ Context Stage2 haddock vanilla}#
#\itab{~~~...}#
#\itab{~~~\_} \bptab{$\rightarrow$ Nothing}#
\end{lstlisting}

\noindent In particular, we can see that \lst'Ghc Stage0' is an external
builder, \lst'Ghc Stage1' is an internal one built from package \lst'ghc'
during \lst'Stage0', \lst'Haddock' is built in \lst'Stage2', etc. There are
27 builders, 16 of which are internal. Furthermore, some builders are
\emph{optional}, e.g., \lst'HsColour', which (if installed) is used to
colourise Haskell code when building documentation.

Each invocation of a builder is fully described by a \lst'Target', which
comprises a build \lst'Context', a \lst'Builder', a list of input files and
a list of output files. 3748 targets are built when building \lst'Stage2' GHC
with documentation (with vanilla and profiled libraries). Consider the following
invocation of \lst'Ghc Stage1' as an example:

\begin{lstlisting}
inplace/bin/ghc-stage1 -O2 -prof -c libraries/base/Prelude.hs
    -o build/stage1/libraries/base/Prelude.p_o
\end{lstlisting}

\noindent The corresponding \lst'Target' is:

\begin{lstlisting}
preludeTarget = Target
#\itab{~~~~\{} \ctab{context} \ttab{= Context Stage1 base profiling}#
#\itab{~~~~,} \ctab{builder} \ttab{= Ghc Stage1}#
#\itab{~~~~,} \ctab{inputs} \ttab{= ["libraries/base/Prelude.hs"]}#
#\itab{~~~~,} \ctab{outputs} \ttab{= ["build/stage1/libraries/base/Prelude.p\_o"] \}}#
\end{lstlisting}

\noindent By examining \lst'preludeTarget' it is possible to compute the full
command line for building \lst'build/stage1/libraries/base/Prelude.p_o':
\begin{itemize}
  \item The builder is \lst'Ghc Stage1'. We lookup the right command
  \lst'inplace/bin/ghc-stage1' with help of \lst'builderProvenance', and use it
  in the following command line template:\vspace{1mm}\\
  \centerline{\!\!\!\!\!\lst'-O2 -c <input> -o <output>'}
  \item The way is \lst'profiling'. We know that we need to add the
  flag \lst'-prof'.$\!\!\!$
  \item We know how to substitute \lst'<input>' and \lst'<output>' in the
  above template.
  \item We ignore \lst'package' since it is not relevant in this particular
  case.
\end{itemize}

\noindent A build system typically contains many such computations (at least one
for each builder) and it is important to provide a terse and readable notation to
describe them. After experimenting with several abstractions, we converged on
\emph{expressions}, as defined in the next subsection.

\subsection{Expressions\label{sec:expressions}}

An expression \lst'Expr a' is a computation that produces a value of type
\lst'Action a' and can read the current build \lst'Target', as shown in
Figure~\ref{fig:abstractions}~(Expressions). For example,
the following expression computes command line arguments for invoking GHC:

\newcommand{\altab}[1]{\hspace{.05\textwidth}\rlap{#1}}

\begin{lstlisting}
ghcArgs :: Expr [String]
ghcArgs = do
    target <- ask
    return $ [ "-O2" ]
        #\hspace{1.5mm}#++ [ "-prof" | way (context target) == profiling ]
        #\hspace{1.5mm}#++ [ "-c", head (inputs target) ]
        #\hspace{1.5mm}#++ [ "-o", head (outputs target) ]
\end{lstlisting}

\subsubsection{Predicates}

The use of conditional \lst'Expr' values is pervasive. In the \lst'ghcArgs' expression
above \lst'-prof' is only applied when profiling. But in fact, the entire expression
is only applicable when using the \lst'Ghc' builder. To make conditionals more concise
we use \emph{predicates} of type \lst'Expr Bool', see
Figure~\ref{fig:abstractions}~(Predicates). In particular, we
use \lst'way :: Way -> Expr Bool' to check which way is currently being built.
For example, predicate \lst'way profiling' returns \lst'True' when the
current target is built using the \lst'profiling' way.

Operator \lst'(?) :: Monoid a => Expr Bool -> Expr a -> Expr a' is a
convenient shortcut for applying a predicate to an expression that computes a monoidal value, such
as \lst'[String]'. For example, the following expression returns
\lst'["-prof"]' when the current target is built the \lst'profiling' way and an
empty list of arguments otherwise:

\begin{lstlisting}
prof :: Expr [String]
prof = way profiling ? return ["-prof"]
\end{lstlisting}

\noindent Expressions computing monoids themselves form a monoid:

\begin{lstlisting}
instance Monoid a => Monoid (Expr a) where
    mempty #\hspace{1.5mm}#= return mempty
    mappend = liftM2 mappend
\end{lstlisting}

\noindent Predicates and monoidal expressions are a powerful combination with
many useful laws that allow us to reason about them:
\begin{enumerate}
  \item \itab{Absorption:} \tab{\lst'p ? mempty === mempty'}
  \item \itab{Distributivity:} \tab{\lst'p ? (e <> f) === p ? e <> p ? f'}
  \item \itab{Conjunction:} \tab{\lst'p ? q ? e === (p &&& q) ? e === q ? p ? e'}
  \item \itab{Disjunction:} \tab{if \lst'p &&& q === False' then} \vspace{1mm}\\
  \lst'    p ? e <> q ? e === (p ||| q) ? e === q ? e <> p ? e'
  \item \itab{Complement:} \tab{\lst'p ? e <> !!!p ? e === e'}
\end{enumerate}

\subsubsection{Extensibility}

All expressions need to be modifiable by users of the build system. We therefore
need to provide a way not only to add new arguments, but also
to modify and remove them. A simple solution is to switch to difference
list expressions, represented by the type \lst'Expr (Diff a)', which is used to
construct values of type \lst'Diff a' with the following monoid instance:

\begin{lstlisting}
instance Monoid (Diff a) where
    mempty #\hspace{24mm}#= Diff id
    mappend (Diff x) (Diff y) = Diff $ y . x
\end{lstlisting}

\noindent The reverse order of function composition \lst'y . x' ensures that
when two \lst'Expr (Diff a)' computations are combined \lst'c1 <> c2', then
\lst'c1' is applied first and \lst'c2' is applied second.

The following functions can be used to append and remove items to/from a
difference list:

\begin{lstlisting}
append :: Monoid a => a -> Expr (Diff a)
append x = return $ Diff (<> x)

remove :: Eq a => [a] -> Expr (Diff [a])
remove xs = return . Diff $ filter (`notElem` xs)
\end{lstlisting}


We are now ready to introduce \lst'Args', a type of expression for
constructing command line arguments in the build system. In addition to the
above generic functions (whose specialised versions are shown in
Figure~\ref{fig:abstractions}), it is equipped with the function
\lst'arg :: String -> Args' for injecting simple \lst'String' arguments into
an expression, e.g. \lst'arg "-prof" :: Args'.

With these abstractions in place, we can construct command
line arguments for GHC as follows:

\begin{lstlisting}
ghcArgs :: Args
ghcArgs = builderGhc ? mconcat
    [ arg "-O2"
    , way profiling ? arg "-prof"
    , arg "-c", arg =<< getInput
    , arg "-o", arg =<< getOutput ]
\end{lstlisting}

\noindent Here \lst'getInput :: Expr FilePath' and
\lst'getOutput :: Expr FilePath' are expressions that check that
\lst'Target.inputs' and \lst'Target.outputs' contain exactly one element and
return it.

The resulting \lst'ghcArgs' expression is terse and readable. All
distracting plumbing details have been abstracted away so that the designers and
users of the build system can focus on what matters.

%\todo{\textbf{Andrey}: Also mention Packages and Ways expressions.}

We compose all command line arguments into a single \lst'args' expression,
applying custom user modifications \lst'userArgs' to the end,
allowing the user to override any default setting:

\begin{lstlisting}
args :: Args
args = mconcat [ ghcArgs, ..., userArgs ]
\end{lstlisting}

The resulting expression is used in the \lst'build' function that is responsible
for building a given \lst'Target':

\begin{lstlisting}
build :: Target -> Action ()
build target@Target {..} = do
    path <- builderPath builder
    need [path]
    checkArgsHash target
    argList <- interpret target args
    cmd [path] argList
\end{lstlisting}

\noindent The \lst'build' function proceeds as follows:
\begin{itemize}
  \item First, \lst'builderPath :: Builder -> Action FilePath' determines the
  path to the builder depending on its provenance and the contents of the
  \lst'system.config' file.
  \item We \lst'need' the builder to make sure it is up-to-date. Some builders
  are built by the build system, e.g. \lst'genprimopcode',
  \lst'ghc-cabal', \lst'Stage1' GHC, so it is important to rebuild them (as well
  as all dependent targets) if needed.
  \item The function \lst'checkArgsHash :: Target -> Action ()' checks whether the
  \lst'target' needs to be rebuilt because the command line computed from
  \lst'args' expression has changed since the previous build. If it has,
  the target is rebuilt even if it is otherwise up-to-date. This step tracks changes both in
  the environment and in the build system itself. We
  track command line hashes using polymorphic dependencies, see~\S\ref{sec:polymorphic}.
  \item We \lst'interpret' the \lst'args' expression w.r.t. to the
  \lst'target', and obtain the list \lst'argList :: [String]' of arguments to be
  passed to the builder. See Figure~\ref{fig:abstractions} for the implementation of
  \lst'interpret'.
  \item Finally, we invoke the builder with appropriate arguments using Shake's
  \lst'cmd' function.
\end{itemize}

\subsection{A simple build system example\label{sec:build-example}}

Figure~\ref{fig:example-abstractions} shows a simple build system that uses the
abstractions introduced in this section.

\newcommand{\tabx}[1]{\hspace{.106\textwidth}\rlap{#1}}
\newcommand{\taby}[1]{\hspace{.074\textwidth}\rlap{#1}}
\newcommand{\tabz}[1]{\hspace{.24\textwidth}\rlap{#1}}

\begin{figure}
\begin{lstlisting}
#\vspace{-7mm}#
#\line(1,0){240}#
#\itab{data Version} \tabx{= Head} \taby{| Release}\hspace{34.5mm}\textbf{\emph{Build~types}}#
#\itab{data Package} \tabx{= Array} \taby{| Base}#
#\itab{data Way} \tabx{= Vanilla} \taby{| Profiling}#

#\itab{data Context} \tabx{= Context Version Package Way}#

#\itab{data Builder} \tabx{= Ghc Version | Ar}#
#\vspace{-5mm}#
#\line(1,0){240}#
args :: Args#\hspace{33mm}\textbf{\emph{Command~line~arguments}}#
args = mconcat
    [ builderGhc ? mconcat
        [ arg "-O2"
        , way Profiling ? arg "-prof"
        , arg "-c", arg =<< getInput
        , arg "-o", arg =<< getOutput ]

    , builder Ar ? mconcat
        [ arg "q"
        , arg =<< getOutput
        , append =<< getInputs ]

    , file "//base/GHC/IO.*" ? arg "-funbox-strict-fields"

    , builderGhc ? package Array ? arg "-Wall" ]
#\vspace{-5mm}#
#\line(1,0){240}#
buildPackage :: Context -> Rules ()#\hspace{20.5mm}\textbf{\emph{Build rules}}#
buildPackage context@Context {..} = do
    path context </> "*" ++ osuf way %> \obj -> do
        let src = obj -<.> "hs"
        deps <- lookupDependencies context obj
        need $ src : deps
        build $ Target context (Ghc version) [src] [obj]

    path context </> "*" ++ asuf way %> \a -> do
        srcs <- lookupSources context
        let objs = [ src -<.> osuf way | src <- srcs ]
        need objs
        build $ Target context Ar objs [a]

path :: Context -> FilePath
path Context#\,#{..} = show#\,#version </> pkgName#\,#package

osuf :: Way -> String
osuf Vanilla   = ".o"
osuf Profiling = ".p_o"

asuf :: Way -> String
asuf Vanilla   = ".a"
asuf Profiling = "_p.a"

lookupDependencies :: Context#\,#->#\,#FilePath#\,#->#\,#Action#\,#[FilePath]
lookupDependencies context src = do ...

lookupSources :: Context -> Action [FilePath]
lookupSources context = do ...
#\vspace{-5mm}#
#\line(1,0){240}#
main :: IO ()#\hspace{59.5mm}\textbf{\emph{Main}}#
main = shake shakeOptions $ do
    for_ [Head, Release] $ \version ->
        for_ [Array, Base] $ \package ->
            for_ [Vanilla, Profiling] $ \way -> do
                let context = Context version package way
                want [#\,#path#\,#context </> "HSlib" ++ asuf#\,#way#\,#]
                buildPackage context
#\vspace{-4mm}#
\end{lstlisting}

\caption{Example of a build system\label{fig:example-abstractions}}
\end{figure}

The build system comprises two packages (\lst'Array' and \lst'Base') that can
be built in two ways (\lst'Vanilla' and \lst'Profiling'), using two versions
of GHC (\lst'Ghc Head' and \lst'Ghc Release') and an archiver tool \lst'Ar'.
Note, the build \lst'Context' is slightly different from GHC, but this does not
prevent us from using the same abstractions.

\newcommand{\checkedbox}{\makebox[0pt][l]{$\square$}\raisebox{.15ex}{\hspace{0.1em}$\checkmark$}}
\newcommand{\uncheckedbox}{\makebox[0pt][l]{$\square$}\raisebox{.15ex}{\hspace{0.92em}}}
\begin{table*}[t]
\centering
\begin{tabular}{p{60mm} || p{50mm} | p{50mm}}
\textbf{Use case}
& \textbf{Old build system} based on Make
& \textbf{New build system} based on Shake
\\
\hline
\textsf{U1:} Fully-featured GHC build
& Everything is built \hfill \checkedbox
& Not all features supported, see~\S\ref{sec:limitations} \hfill \uncheckedbox
\\
\textsf{U2:} Clean build
& Everything is built \hfill \checkedbox
& Everything is built \hfill \checkedbox
\\
\textsf{U3:} Zero build \hspace{6.4mm}
& Nothing is rebuilt \hfill \checkedbox
& Nothing is rebuilt \hfill \checkedbox
\\
\hline
\textsf{U4:} Touch: \hspace{10.2mm}\textsf{libraries/base/Prelude.hs}
& \textsf{Prelude.o}, \textsf{base} library, and all \hfill \uncheckedbox
\newline dependent binaries are rebuilt
& Nothing is rebuilt \hfill \checkedbox
\\
\textsf{U5:} Add comment: \textsf{libraries/base/Prelude.hs}
& \textsf{Prelude.o}, \textsf{base} library, and all \hfill \uncheckedbox
\newline dependent binaries are rebuilt
& Only \textsf{Prelude.o} is rebuilt \hfill \checkedbox
\\
\textsf{U6:} Modify code: \hspace{1.75mm}\textsf{libraries/base/Prelude.hs}
& \textsf{Prelude.o} and all its dependencies \hfill \checkedbox \newline
are rebuilt
& \textsf{Prelude.o} and all its dependencies \hfill \checkedbox \newline
are rebuilt
\\
\textsf{U7:} Add comment: \textsf{utils/ghc-cabal/Main.hs}
& Almost everything is rebuilt \hfill \uncheckedbox
& All \textsf{ghc-cabal} rules are rerun \hfill \uncheckedbox
\\
\textsf{U8:} Modify code: \hspace{1.75mm}\textsf{utils/ghc-cabal/Main.hs}
& Almost everything is rebuilt \hfill \uncheckedbox
& Only the targets affected by the \hfill \checkedbox \newline change are rebuilt
\\
\hline
\textsf{U9:} $\textit{~~}$Modify the build system: pass \textsf{-O2} when
\newline $\textit{~~~~~~~~~}$compiling Stage2 GHC
& Nothing is rebuilt \hfill \uncheckedbox
& Stage2 GHC and its dependencies \hfill \checkedbox \newline are rebuilt
\\
\textsf{U10:} Modify the build system without changing \newline
$\textit{~~~~~~~~~}$command line arguments of build tools
& Nothing is rebuilt \hfill \uncheckedbox
& Nothing is rebuilt \hfill \uncheckedbox
\\
% \hline
% \textsf{U10:} Switch to a \textsf{git} branch and back
% & Some files are rebuilt \hfill \uncheckedbox
% & Nothing is rebuilt \hfill \checkedbox
% \\
\textsf{U11:} Change path to \textsf{gcc}
& Everything is rebuilt \hfill \uncheckedbox
& Files depending on \textsf{gcc} are rebuilt \hfill \checkedbox
\\
\end{tabular}
\caption{Comparison of GHC build systems on common use cases. Checkmarks
\checkmark indicate desired behaviour.}
\label{tab:use-cases}
\end{table*}

Build rules are generated by the \lst'buildPackage' function, which given a
build \lst'Context', describes how to compile a Haskell source file and build a
library by archiving the obtained object files. \lst'buildPackage' relies on
several helper functions, which define the \lst'path' to build artefacts,
set way-specific extensions for object and archive files, lookup
dependencies of a source file, and compute a list of source files for a given
library. Command line arguments are specified as a single expression~\lst'args',
which makes use of package-, way-, builder-, and file-specific arguments.

The \lst'main' function is straightforward: for each possible \lst'Context'
we request the corresponding library to be built using \lst'want', as well as
generate necessary rules by calling \lst'buildPackage'. If we run the build
system it will build 8 libraries and all associated object files.

