\section{Introduction}

A build system is a critical component of any software project.
It is responsible for collecting all source code written in various languages,
compiling it, linking with standard libraries, running automated tests, and
producing executable artefacts -- end software products. Build systems of large
software projects, such as the Glasgow Haskell Compiler~(GHC), are complex
engineering artefacts. Their complexity stems from several factors: they evolve everyday as
new features are continuously being added by engineers spread across multiple
continents, they must support a variety of target hardware configurations, and
last but not least they operate under extreme correctness and performance
requirements because they stand on the critical path between the development of
a new feature and its deployment into production. 

~\\
\textbf{TODO}:
\begin{itemize}
  \item Motivating examples of mission-critical build systems, perhaps, from
  Standard Chartered and/or Facebook?
  \item Brief introduction to Shake, a Haskell library for writing build systems
  that is a key component to our work.
  \item How does this paper improve on the state-of-the-art?
  \item List of specific contributions.
\end{itemize}

We use \cite{mitchell:shake}.