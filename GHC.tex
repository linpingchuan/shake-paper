\section{Shaking up GHC\label{sec:ghc}}

In this section we report on our experience of applying the techniques presented
so far to building a large-scale software project: the Glasgow Haskell Compiler.
We implemented\footnote{\urlstyle{sf}\url{https://github.com/snowleopard/hadrian}}
a new build system for GHC from scratch using Shake and our build abstractions
from~\S\ref{sec:abstractions}. The new build system does not yet implement the
full functionality of the old build system, but we are currently addressing
remaining limitations; nothing presents any new challenges or requires changes
to the build infrastructure.

% \subsection{Current status and limitations\label{sec:limitations}}
%
% We implemented a new build system for GHC from scratch using Shake and our build
% abstractions from~\S\ref{sec:abstractions}. The current implementation can build
% \lst'Stage2' GHC, but has the following limitations:
% \begin{itemize}
%   \item We only build \lst'vanilla' and \lst'profiling' way.
%   \item We reuse GHC testing infrastructure of the old build system.
%   \item Only HTML documentation can be built.
%   \item Not all build flavours and command line flags are supported.
%   \item Cross-compilation is not implemented.
%   \item Installation or binary/source distribution are not supported.
% \end{itemize}
%
% \noindent We intend to fix the above in the near future; nothing presents any
% new challenges or requires changes to the build infrastructure.

\subsection{Qualitative Analysis\label{sec:use-cases}}

In this section we discuss several use cases of the GHC build system, which
are fairly typical for build systems in general. Table~\ref{tab:use-cases}
lists use cases \lst'U1-U11' highlighting differences between the old and the
new build systems. Below we go through some of the use cases in more detail.
See~\S\ref{sec:benchmarks} for performance comparison.

\lst'U1-U3' are simplest use cases. For the sake of fairness we start with
\lst'U1', where the old build system reigns over our current implementation due
to the aforementioned limitations. When unsupported features are not used
(\lst'U2'), the new build system successfully builds all expected targets.
Running a build system twice in a row must be equivalent to only running it
once; the second build must do nothing, hence the name \emph{zero
build} (\lst'U3'). The new build system works as expected, and is faster
than the old one (see \S\ref{sec:benchmarks}).

% The time it takes represents internal overheads of the build system:
% scanning the file system, reading the database, etc.

In \lst'U4-U6' we modify \lst'libraries/base/Prelude.hs' and rebuild GHC.
If we \lst'touch' the file (\lst'U4'), i.e. change only its modification
time, the new build system rebuilds nothing, as desired. The old build system rebuilds
\lst'Prelude.o', the \lst'base' library, and all dependent binaries, such as
\lst'Stage2' GHC. This use case commonly occurs when switching \lst'git'
branches, as explained in~\S\ref{sec:file-contents}, or whenever a user changes
a file, but then decides to undo the changes. In \lst'U5' we add comments,
forcing the new build system to recompile \lst'Prelude.hs'. It then notices the
object code is unchanged, and stops: there is nothing else to be done. The old
build system continues to rebuild the \lst'base' library and dependent binaries,
which is unnecessary. In \lst'U6' the modification of \lst'Prelude.hs' leads to
changes in \lst'Prelude.o', which causes all dependencies to be rebuilt. Both
build systems handle this case correctly.

\lst'U7-U8' are similar, but we now modify sources of the \lst'ghc-cabal' build
tool. The old build system rebuilds almost everything in both cases,
which is unnecessary. Rebuilds are caused by rerunning the updated \lst'ghc-cabal'
binary, which changes modification time of generated \lst'package-data.mk'
files. The lack of polymorphic dependencies means we have to depend on the whole
file when using Make, therefore even if only one field in a
\lst'package-data.mk' file is changed, e.g. \lst'CC_OPTS', we end up
rebuilding everything, not only C compilation rules that depend on
\lst'CC_OPTS'. The new build system uses Shake's polymorphic
dependencies~\S\ref{sec:polymorphic} to avoid such unnecessary rebuilds. However,
\lst'U7' behaviour is still suboptimal: \lst'ghc-cabal' rules
are rerun, because GHC currently produces non-deterministic output
(\lst'ghc-cabal''s binary is changed).

In \lst'U9' we modify the build system itself by changing command line
arguments for one of the build files. The old build system rebuilds nothing,
as Make does not track such changes. The new build system correctly reruns all
affected rules. We currently only track command line arguments, therefore other,
more subtle modifications of the build system (\lst'U10') go unnoticed, for
example, adding a new \lst'need' dependency does not cause a rebuild.
In \lst'U11' we modify the build environment, by changing the path to \lst'gcc'
in the configuration file. As in \lst'U7-U8', the old build system rebuilds
almost everything since depending on a single configuration setting is not
supported. The new build system correctly reruns only affected rules.
%\todo{\textbf{Andrey}: Add \lst'U12', edit cabal.in case.}

% Working with multiple \lst'git' branches is a problematic use-case for build
% systems based on Make (as discussed in the second bullet point of
% \S\ref{sec:file-contents}). In \lst'U10' we conduct the following simple
% experiment:
%
% \begin{lstlisting}
% $ build
% $ git checkout -b test
% $ echo -- comment >> libraries/base/Prelude.hs
% $ git checkout master
% $ build
% \end{lstlisting}
%
% We expect the second \lst'build' to do nothing, since the \lst'master'
% branch has not changed. However, \lst'git checkout master' changes modification
% time of all files that were updated between branches, forcing the new
% build system to rerun all dependent rules (as in \lst'U4'). Shake's ability to
% track file contents in addition to modification time~\S\ref{sec:file-contents}
% allows us to avoid such unnecessary rebuils.

In summary, the new build system correctly handles most use cases, whereas
the old one performs a lot of unnecessary rebuilds in many cases.

\vspace{-1mm}
\subsection{Quantitative Analysis: Benchmarks\label{sec:benchmarks}}

When building from scratch, ignoring the initial \lst'boot'/\lst'configure'
steps which are shared, the old build system takes 1266 seconds on Windows and
649 seconds on Linux; while the new build system takes 737 seconds on Windows and
578 seconds on Linux. These were tested in as similar configurations as we could
manage (by disabling all the features the new system does not support),
but due to the complexity of the build systems, there are almost certainly
minor differences. For the zero build, the old build system takes 12.3
seconds on Windows and 2.2 seconds on Linux; while the new one takes
2.1 seconds on Windows and 2.0 seconds on Linux. All measurements were collected
using \lst'-j4'.

Since Shake and Haskell both provide profiling and analysis tools, we have
already used these features to optimise the new build system, resulting in
modest gains so far and several opportunities we have yet to exploit. Looking at
the current full Windows build time of 737 seconds, the longest single task is
building the GMP library (315 seconds), and the total time of single-threaded
computation is 2206 seconds, of which 2099 is calling out to external processes.
The critical dependency chain has 378 steps in it, and requires 463 seconds -- a
lower bound on the clean build time even with an unlimited number of processors.
