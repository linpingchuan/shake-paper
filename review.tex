\section{Related Work\label{section-review}}

This paper is about writing build systems at scale, the most complete modern advice on developing such systems comes from \citet{smith:software_build_systems}. When
\citet{mcintosh:build_maintenance_effort} studied software maintenance they
found that build systems can take up to 27\% of the
development effort, and that improvements to the build system rapidly paid off.
Recently \citet{martin:make_it_simple} surveyed which Make features are used,
and then \citet{martin:maintenance_complexity_makefiles} classified them by
complexity -- unsurprisingly they found that as Makefiles grow, their complexity
increases, and that hand written Makefiles require most complex features. In
the remainder of this section we focus on some of the features found in other
build systems which could be useful at scale.

\subsection{Embedded language}

A build system can either be specified using structured metadata, e.g. Bazel
\cite{bazel}, or embedded into a standard programming language -- for example
SCons in Python \cite{scons}, Pluto in Java \cite{pluto} and Jenga in OCaml
\cite{jenga}. For complex bespoke build systems, embedding into a language
allows both complex operations (\S\ref{sec:real_code}) and better abstractions
(\S\ref{sec:abstractions}) -- essentially allowing us to write most of our build
system in a domain language tailored to our specific project.

Even sticking to Haskell as the embedded language, there are a surprisingly
large number of libraries implementing a dependency aware build system -- we
know of eleven in addition to Shake (Abba, Blueprint, Buildsome, Coadjute, Cake
$\times$ 2, Hake, Hmk, Nemesis, OpenShake and Zoom). Of these, the two Cake
libraries and OpenShake are based on an early presentation of the principles behind Shake.

\subsection{Advanced dependencies}

We have found that while powerful dependencies might only be used in a few
places, if they are missing the workarounds can be pervasive
(\S\ref{sec:dynamic-deps}). A few build systems support resources, e.g.
Ninja \cite{ninja}, and several support monadic dependencies, e.g. Redo
\cite{redo}, Jenga, Pluto, SCons. A few build systems directly support
dependency features more powerful than Shake, e.g. Pluto supports rules
that run until a fixed-point is reached and rules whose output filename is not known in
advance. These features can be encoded in Shake, but are not present natively.

\subsection{Automatic dependency management}

In both Shake and Make, all dependencies must be declared explicitly. However,
in build systems such as Tup \cite{tup} and Buildsome \cite{buildsome}, some
dependencies are automatically captured by monitoring program execution, albeit
only \emph{after} the dependency has been used (like \lst'needed' in
\S\ref{sec:needed}). The Fabricate tool \cite{fabricate} takes a unique approach
to defining build systems, providing a series of steps that run sequentially,
but are skipped if their automatically-detected inputs have not changed.
Unfortunately no cross-platform APIs are available to detect used dependencies,
so such tools are all limited in which platforms they support.

\subsection{Build clusters}

The build systems Bazel and Buck \cite{buck} are used at Google and Facebook
respectively, both operating at sizes significantly beyond that of the GHC build
system (reportedly billions of lines of code). Both systems take a metadata
approach, with various rule types baked in. As an example, the \lst'cxx_binary'
rule builds a C/C++ binary given a list of source files and dependencies,
taking care of suitable build flags and conventions, much like a very sophisticated
version of \lst'buildPackage' from~\S\ref{sec:build-example}.

The disadvantage of such an approach is that the available rules are fixed,
making it difficult to encode something like a bootstrapping compiler.
Generating build metadata is not really supported -- a problem typically solved by
committing the generated files to version control. Both tools also support build
clusters, which build code once and share the resulting objects to everyone
without recompiling them locally -- an essential feature at such scales.
