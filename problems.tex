\section{Challenges of large-scale build systems\label{sec:challenges}}

Many existing build systems work well for small projects, or
projects that follow a common pattern.  For example, building a single
executable or library from single-language source files is a
well-supported in virtually all build systems.  A
lot of projects fit into this category, and so never run into the
limits of existing build systems.

Things start to get hairy in big projects, when complexities such as these
show up:
\begin{itemize}
\item The project has a large number of components (libraries or
  executables), using multiple languages, with complex
  interdependencies.  Executables depending on libraries, for example,
  or tools generating inputs for other parts of the build system.
\item Many components follow similar patterns, so there is a need for
  abstractions that allow common functionality to be shared between
  different parts of the build system.
\item Parts of the build system are not static but are generated,
  perhaps by running tools that are built by the same build system.
\item The build system has complex configuration, with aspects of its
  behaviour being controlled by a variety of different sources:
  automatic platform-specific configuration, command-line options,
  configuration files, and so on.
\end{itemize}
The GHC build system includes all of the above. To illustrate the
consequences, here is one rule from GHC's current build system, which
is based on Make:
\begin{lstlisting}[basicstyle=\footnotesize\sffamily,escapeinside={(*}{*)}]
$1/$2/build/%.$$($3_osuf) : \
    $1/$4/%.hs $$(LAX_DEPS_FOLLOW) \
    $$$$($1_$2_HC_DEP) $$($1_$2_PKGDATA_DEP)
  $$(call cmd,$1_$2_HC) $$($1_$2_$3_ALL_HC_OPTS) \
    -c $$< -o $$@ \
    $$(if $$(findstring YES,$$($1_$2_DYNAMIC_TOO)), \
       -dyno $$(addsuffix .$$(dyn_osuf),$$(basename $$@)) )
  $$(call ohi(*-*)sanity(*-*)check,$1,$2,$3,$1/$2/build/$$*)
\end{lstlisting}
Yes, there are four dollar signs in a row!  If it wasn't so tragic,
impenetrable code like this would be hilarious.  This kind of thing
goes on for \emph{thousands} of lines.  The result is a nighmare to
understand, maintain and modify.

Perhaps GHC's implementors just aren't very clever?  Maybe so, but
they did at least try hard. The current build system is the
fourth major iteration, as they struggled to find good solutions to the
problems that arose:

\begin{itemize}
\item The first incarnation of the build system used \lst'jmake', a
  system inspired by the X Consortium's \lst'imake'.  This was
  essentially ordinary Make together with the C preprocessor to
  allow the use of macros in \lst'Makefile's.  The macro layer partly
  solves the problem of needing to share functionality between
  different parts of the build system.

\item The C preprocessor was somewhat painful to use with Make.
  The syntax was difficult to get right, and the need to constantly
  \lst"make Makefiles" when working on the build system was tiresome.  GNU
  Make came along which had built-in support for \lst'include' files and
  other features, meaning the GHC build system could do away with the C
  preprocessor. The build system was rewritten to use GNU Make, and
  simulated macros with include files (GNU Make didn't have macros at
  the time).

\item Next there were several large changes to the GHC build system
  that didn't amount to complete rewrites.  First, the build system started
  to build multiple bootstrap \emph{stages} in a single build tree; that is,
  build the compiler (stage~1) and then build the compiler again using the
  stage~1 compiler (stage~2). Previously this process had required two separate builds.

  Around this time the library ecosystem of Haskell was exploding, and
  the Cabal build system for Haskell libraries emerged. The GHC build system was
  integrated with Cabal to avoid duplicating Cabal metadata or build logic for
  the libraries that were part of the GHC.
%   We wanted to
%   integrate Haskell libraries with Cabal build systems into the GHC
%   build, but without duplicating the Cabal metadata or build logic for
%   those packages.  So we integrated our build system with Cabal,

\item The build system in its current form had grown unwieldy, and had
  many idiosyncrasies. The root of many of the problems was that the
  build system was constructed as a set of Makefiles that recursively
  invoked each other. Recursive Make is considered harmful
  for very good reasons~\cite{miller:recursive_make}; it is not possible to
  accurately track dependencies when the build system is constructed
  of separate components that invoke each other.

  So in the next rewrite of the build system it was made non-recursive.
  However, in doing so, GNU Make was stretched to its absolute
  limits, as the example above illustrates. Every part of the rule is
  there for a good reason, but the cumulative effect of solving all
  the problems that arise in a complex build system is impenetrable.
\end{itemize}
\noindent
By carefully employing a set of
idioms\footnote{\urlstyle{sf}\url{https://ghc.haskell.org/trac/ghc/wiki/Building/Architecture}}
(such as the prefix idiom \lst'$1_$2_' above, which is explained
in~\S\ref{sec:pattern-rule-language}), it was possible to construct a
non-recursive build system for GHC in GNU Make. It works surprisingly well, and
by virtue of being non-recursive it tracks dependencies accurately; but, it is
almost impossible to understand and maintain. So while it is \textit{possible}
to develop large scale build systems using Make, it is clear that GHC developers
have gone far beyond Make's scalability limits.

% \begin{enumerate}
% \item The programming language is complex. This macro expands through multiple
% levels, so \lst"$$$$($1_$2_HC_DEP)" looks up in the environment two stages
% later.
% \item The dependency on \lst"$1_$2_PKGDATA_DEP" is only to reduce parallelism,
% to ensure too many commands do not run in parallel.
% \end{enumerate}

% \subsection{The GHC build system}
%
% Build systems need to evolve quickly with the projects they support. While it is
% \textit{possible} to develop large scale systems in \make, it is not
% \textit{pleasant} -- resulting in heroic efforts where straight-forward
% simplicity should be preferred.
%
% The GHC build system is a complex multi-language build system. The system is a
% bootstrapping compiler, where the GHC compiler is built using a system compiler,
% then recompiled using built compiler and recompiled system libraries. This
% pattern naturally contains repeated patterns, but capturing them in \make{} is
% hard.
%
% We focus on the GHC build system for two reasons. Firstly, it is the coal-face
% at which many of us work, so improvements to it will bring real improvements to
% our daily development. Secondly, the GHC build system has many complex features
% that test the limits of existing build systems:
%
% \begin{itemize}
% \item GHC is cross-language, including large amounts of both C and Haskell code.
% It also generates user manuals from docbook, which can be viewed as another
% language with unique build/dependency patterns.
% \item GHC is a bootstrapping compiler with stages. It first builds a compiler
% using the system compiler, then uses that new compiler
% \item The GHC build system necessarily integrates with other build systems,
% including \make{} (for building libgmp) and \cabal{} (for building/registering
% Haskell libraries).
% \item It generates files, e.g. \texttt{compiler/stage1/ghc\_boot\_platform.h}
% that \texttt{\#define}'s various platform-specific constants used throughout the
% build system.
% \item It is cross-platform, working on Windows, Linux, Mac, iOS, Android,
% Solaris, BSD flavours etc.
% \item The system is in constant flux as new features are added to the system.
% \end{itemize}
%
% Writing such a build system remains a challenging engineering undertaking, but
% one we now hope not to have to repeat. This is the fifth version, and interestingly, perhaps the most user friendly. Applying all the guidelines in Recursive Make Considered Harmful we have gone through the following stages.
%
% \begin{enumerate}
%   \item jmake - cpp + make clone of X11 imake
%   \item GNU make - drop cpp, recursive make, manual stages
%   \item GNU make - no manual stages, cabal integration, first
%    use of macros
%    \item GNU make - current iteration, non recursive, cite Recursive
%      Make Considered Harmful here, extensive use of macros,
%      building abstractions in make
% \end{enumerate}
