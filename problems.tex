\section{Challenges of large-scale build systems\label{sec:challenges}}

Writing a large build system using make is challenging. The GHC build system has been reweritten 4 times, and even after that we've ended up with:

\begin{lstlisting}[basicstyle=\footnotesize\sffamily,escapeinside={(*}{*)}]
$1/$2/build/%.$$($3_osuf) : $1/$4/%.hs $$(LAX_DEPS_FOLLOW) \
    $$$$($1_$2_HC_DEP) $$($1_$2_PKGDATA_DEP)
  $$(call cmd,$1_$2_HC) $$($1_$2_$3_ALL_HC_OPTS) -c $$< -o \
    $$@ $$(if $$(findstring YES,$$($1_$2_DYNAMIC_TOO)),-dyno \
    $$(addsuffix .$$(dyn_osuf),$$(basename $$@)))
  $$(call ohi(*-*)sanity(*-*)check,$1,$2,$3,$1/$2/build/$$*)
\end{lstlisting}

This code shows some of the problems of a large-scale build system:

\todo{list all the problems, mostly from the solutions section}

\begin{enumerate}
\item The programming language is complex. This macro expands through multiple levels, so \lst"$$$$($1_$2_HC_DEP)" looks up in the environment two stages later.
\item The dependency on \lst"$1_$2_PKGDATA_DEP" is only to reduce parallelism, to ensure too many commands do not run in parallel.
\end{enumerate}

\subsection{The GHC build system}

Build systems need to evolve quickly with the projects they support. While it is
\textit{possible} to develop large scale systems in \make, it is not
\textit{pleasant} -- resulting in heroic efforts where straight-forward
simplicity should be preferred.

The GHC build system is a complex multi-language build system. The system is a
bootstrapping compiler, where the GHC compiler is built using a system compiler,
then recompiled using built compiler and recompiled system libraries. This
pattern naturally contains repeated patterns, but capturing them in \make{} is
hard.

We focus on the GHC build system for two reasons. Firstly, it is the coal-face
at which many of us work, so improvements to it will bring real improvements to
our daily development. Secondly, the GHC build system has many complex features
that test the limits of existing features:

\begin{itemize}
\item GHC is cross-language, including large amounts of both C and Haskell code.
It also generates user manuals from docbook, which can be viewed as another
language with unique build/dependency patterns.
\item GHC is a bootstrapping compiler with stages. It first builds a compiler
using the system compiler, then uses that new compiler
\item The GHC build system necessarily integrates with other build systems,
including \make{} (for building libgmp) and \cabal{} (for building/registering
Haskell libraries).
\item It generates files, e.g. \texttt{compiler/stage1/ghc\_boot\_platform.h}
that \texttt{\#define}'s various platform-specific constants used throughout the
build system.
\item It is cross-platform, working on Windows, Linux, Mac, iOS, Android,
Solaris, BSD flavours etc.
\item The system is in constant flux as new features are added to the system.
\end{itemize}

Writing such a build system remains a challenging engineering undertaking, but
one we now hope not to have to repeat. This is the fifth version, and interestingly, perhaps the most user friendly. Applying all the guidelines in Recursive Make Considered Harmful we have gone through the following stages.

\begin{enumerate}
  \item jmake - cpp + make clone of X11 imake
  \item GNU make - drop cpp, recursive make, manual stages
  \item GNU make - no manual stages, cabal integration, first
   use of macros
   \item GNU make - current iteration, non recursive, cite Recursive
     Make Considered Harmful here, extensive use of macros,
     building abstractions in make
\end{enumerate}
