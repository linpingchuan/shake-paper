\section{Quick wins: from Make to Shake\label{sec:solutions}}

The GHC build system stretches Make beyond its limits.
In this section we illustrate, from our experience with GHC's build system,
some of these challenges, how they can be solved in our new system.
These problems can be divided into two groups: those caused by the
Make language (solved using Haskell); and those caused by a lack of expressive
power when describing dependencies (solved using Shake).
Building on these ``quick wins'', we show how to structure a
large build system in~\S\ref{sec:abstractions}.

\subsection{Variables}

Make's program state involves a global namespace of mutable string variables that are
spliced into the program. This model naturally causes challenges:

\begin{itemize}
\item Since variables live in a single global namespace, there is limited
encapsulation and implementation hiding.
\item Since variables are strings, arrays and associative maps are typically
encoded using computed variable names, which is error-prone.
\item Since variable references are spliced into the Makefile contents
and interpreted, certain special characters can cause problems --
notably space (which splits lexemes) and colon (which declares rules, but is
also found in Windows drive letters).
\end{itemize}

\noindent By using a build system embedded in Haskell, or indeed any other
modern programming language, most of these issues are eliminated. An alternative solution is to generate the Makefile, either using a tool such as Automake or a full programming language. However, a generator cannot interact with the
build system after generation, resulting in problems with dynamic dependencies
(\S\ref{sec:dynamic-deps}).

\subsection{Macros\label{sec:pattern-rule-language}}

Consider the following Make rule:

\begin{lstlisting}
%.o : %.hs
    ghc $HC_OPTS $<
\end{lstlisting}

\noindent It tells Make that object files \lst"*.o" are produced from Haskell
source files \lst"*.hs" by compiling them with \lst"ghc" command invoked with
\lst"HC_OPTS" arguments. The notation is terse and works well for this example.
Unfortunately, this simple formulation does not scale:

\begin{itemize}
\item What if we want the rule to match \lst"foo.o" and \lst"bar.o", but not
\lst"baz.o"? It is impossible to do any non-trivial computation -- we are forced
to rely on patterns whose expressive power is limited.
\item What if \lst"HC_OPTS" should depends on the file being compiled?
\end{itemize}

The standard Make approach to solve these problems is to use \emph{macros}.
As an example, here is a highly simplified version of the macro from~\S\ref{sec:challenges}:

%define hs-compile =
%endef
\begin{lstlisting}
$1/$2/build/%.o : $1/$4/%.hs
    ghc $$($1_$2_$3_HC_OPTS) -c $$< -o $$@
\end{lstlisting}

\noindent As before, the rule is responsible for compiling a Haskell source
file into an object file. The arguments to the macro are available as \lst'$1'
to \lst'$4'. We have solved the two problems above, but inelegantly:

\begin{itemize}
\item To make up for weak pattern matching, we use macros to generate a separate
rule for each \lst'$1/$2/$4' combination.
% The restriction to certain directories is solved by using the macro to
% generate rules, and calling that macro at all relevant matches. We are
% duplicating the rule to make up for weak pattern matching, and are still quite
% restricted in what can be expressed.
\item The dependence of the command line arguments on the file is solved using a
computed variable name, which is dereferenced using \lst'$$(...)' to ensure the value is
expanded when the macro is \emph{used}, not when it is \emph{defined}. However,
certain variables are present in positions where the value is unavailable when
first called, requiring \lst'$$$$(..)' to further delay expansion. Alas, it is
far from obvious when extra delaying is required.
\end{itemize}

\noindent Using Shake, the simplest variant looks slightly more complex, because
it must be expressed in Haskell syntax:

\begin{lstlisting}
"*.o" %> \out -> do
    let hs = out -<.> "hs"
    need [hs]
    cmd "ghc" hc_opts hs
\end{lstlisting}
\noindent
However, as the complexity grows, the build system scales properly. Making
\lst"hc_opts" depend on the Haskell file requires the small and obvious change:

\begin{lstlisting}
cmd "ghc" (hc_opts hs) hs
\end{lstlisting}

\noindent Namely, we turn \lst'hc_opts' from a constant to a function. To use
richer pattern matching we can drop down to a lower-level Shake operation. In
Shake \lst"%>" is itself defined in terms of \lst"?>":

\begin{lstlisting}
pattern %> act = (pattern ?==) ?> act
\end{lstlisting}
\noindent
Wildcard pattern matching is just a special case, and we can use an
arbitrary predicate to exert more precise control over what matches.
%In both cases the generality of higher-order functions solves the problem.

An alert reader may have noticed that \lst'$1_$2_$3_HC_OPTS' refers to
\lst'$3', which is not related to the file being built. Indeed, the
macro argument \lst'$3' specifies \emph{how} the file must be built. 
The next subsection discusses this pattern and associated challenges.

\subsection{Computing command lines\label{sec:command-lines}}

In any large-scale build system, there are rules that say how to build targets,
and then there is a very long list of special cases, listing targets that need
to be treated specially.
For example, the GHC build system generates different compilation rules
for each combination of package, compiler version, and build way
(in~\S\ref{sec:ghc-context} we refer to such combinations as \emph{build
contexts}). The build rules differ in several aspects, and in particular they
use different command line arguments. The following snippet shows how command
line arguments \lst'$1_$2_$3_HC_OPTS' are computed in the GHC build system:

\begin{lstlisting}
WAY_p_HC_OPTS = -static -prof
...
base_stage1_HC_OPTS = -this-unit-id base
...
$1_$2_$3_HC_OPTS = $$(WAY_$3_HC_OPTS) \
#\hspace{34.4mm}#$$($1_$2_HC_OPTS)\
#\hspace{34.4mm}\emph{\textrm{... plus 30 more configuration patterns}}#
\end{lstlisting}

\noindent This says that when building targets with GHC in the \emph{profiling
way} \lst'p' (i.e. with the profiling information enabled), we need to add
\lst'-static -prof' to the command line. Furthermore, when we compile targets
from the \lst'base' library using the \lst'Stage1' GHC, we need to add
\lst'-this-unit-id base' to the command line. 

\lst'WAY_p_HC_OPTS = -static -prof' is concise, but is inflexible. For example,
if we do not want to pass \lst'-prof' flag when building a particular file, we
have to either add a new component to the variable name, e.g.
\lst'WAY_p_file_HC_OPTS', or conditionally filter out the \lst'-prof' flag from
\lst'HC_OPTS' after it is constructed and the file being built is known. Both
approaches are not scalable; consequently, much of the complexity of the GHC
build system is caused by computing command lines. This complexity is
unnecessary and can be tackled using high-level abstractions readily available
in Haskell. We elaborate on this in~\S\ref{sec:expressions}, where we design a
DSL for succinct and type-safe computation of build command lines.

% The approach allows us to write the following:
% \begin{lstlisting}
% hc_opts = builder Ghc ? mconcat
%     [ way profiling ? append ["-static", "-prof"]
%     , ...
%     , file "//special.o" ? remove ["-prof"]
%     , ... ]
% \end{lstlisting}
% 
% \noindent Here the first line tells us that \lst'hc_opts' is applicable only
% when running GHC; the second line appends \lst'-static -prof' in profiling way; and
% finally flag \lst'-prof' is removed when we build \lst'special.o'. 
% 
% Large-scale software projects may contain thousands of such special cases
% defined throughout the build system, possibly by different authors. Determining
% the provenance of each argument in the final command line is therefore very
% hard, and is generally not addressed in build systems. We designed our DSL with
% the provenance tracking in mind, as discussed in~\S\ref{sec:expressions}.

% The question of provenance is particularly important in large multi-author build
% systems, but generally not something that is addressed anywhere - one of our key
% inovations in \S?. The encapsulation is provided by Haskell modules. The lack of
% freedom is provided by strong name binding. The problems related to `:' go away
% by manipulating values rather program text.

% As we develop more Shake-based build systems, patterns have started to emerge.
% Typically 90\% of a build system can be captured in some simple DSL, taking
% advantage of conventions, and 10\% cannot. As an example, a large build system
% might build 100 C++ libraries, each with similar flags and file layout, but
% taking the source files from different directories. It may also minify a
% Javascript file, and build an installer -- both one-off tasks. Using a
% fully-powerful system such as Shake, it is possible to engineer robust
% abstractions, and then define the majority of the build system using only these
% abstractions. The end result is that most edits to the build system involve only
% the DSL, and can be performed by a large number of individuals.
% 
% After dealing with the DSL, there are usually a few pieces left over, and these
% can be implemented in Shake, as normal. Thanks to the power of Shake you can
% interpret the DSL and combine it with the custom pieces. In our experience by
% providing an \emph{escape hatch} where fully-powerful code an be expressed it
% removes the temptation to shoehorn more advanced features in the DSL, and thus
% avoids turning it into an ad-hoc scripting language. Instead, should enough
% pieces be required in the escape hatch, they can be abstracted in the
% traditional ways - perhaps even combining two DSLs in one build system.

\subsection{Rules with multiple outputs\label{sec:multiple-outputs}}

Given a Haskell source file \lst'Foo.hs', GHC compilation produces both an
interface file \lst'Foo.hi' and an object file \lst'Foo.o'. The typical way of
expressing multiple outputs in Make is to use two rules, one depending
on the other. For example:

\begin{lstlisting}
%.o : %.hs
    ghc $HC_OPTS $<
#\vspace{-2mm}#
%.hi : %.o ;
\end{lstlisting}
\noindent The second rule is a no-op: it tells Make that an interface file can be
produced simply by depending on the corresponding object file. Alas, this
approach is fragile: if we delete the interface file, but leave the object
file intact, Make will not rerun the \lst'*.o' rule, because the object file is
up-to-date. Consequently, the build system will fail to restore the deleted
interface file.

In Shake we can express this rule directly using the operator \lst'&%>',
which defines a build rule with multiple outputs:

%(it actually turns out to be a
%consequence of the more powerful dependency model):

\begin{lstlisting}
["*.o", "*.hi"] &%> \[o, hi] -> do
    let hs = o -<.> "hs"
    need [hs]
    cmd "ghc" hc_opts hs
\end{lstlisting}

\subsection{Reducing concurrency\label{sec:ghc-pkg-db}}

As discussed in~\S\ref{sec:concurrency-reduction}, it is sometimes necessary to
reduce concurrency in a build system. As an example, GHC packages need to
be registered by invoking the \lst'ghc-pkg' utility. This utility mutates the
global state (package database) and hence at most one package can be registered
at a time, or the database is corrupted.

In Make, the solution is to introduce fake \emph{concurrency
reduction} dependencies. In GHC's build system there are 25
packages that require registration,
and in the old build system they all depended on each other in a chain, to
ensure no simultaneous registrations occur. This solution works, but is
fragile (easy to get wrong) and inefficient (reduces available parallelism).

In Shake, the solution is to use the \emph{resources} feature
(\S\ref{sec:concurrency-reduction}):

\begin{lstlisting}
packageDb <- newResource "package-db" 1

action $ withResource packageDb 1 $ cmd "ghc-pkg" ...
\end{lstlisting}

This snippet declares a global resource named \lst"packageDb" with quantity 1,
then later calls \lst"withResource" asking for a single quantity of the resource before running the
\lst'ghc-pkg' utility. Provided all \lst'ghc-pkg' calls are suitably wrapped,
we will never run two instances simultaneously. Furthermore, thanks to the
availability of functions, we can abstract a function that both executes
\lst'ghc-pkg' and applies the resource.
% Note that build systems already schedule tasks to satisfy the implicit thread
% resource, so having user defined resources is an obvious generalisation.

\subsection{Dynamic dependencies\label{sec:dynamic-deps}}

Make, in common with many other build systems, works by
constructing a dependency graph and then executing it. This approach
makes it possible to analyse the graph ahead of
time, but it is limiting in some fundamental ways.  Specifically,
it is common that parts of the dependency graph can be known only after
executing some other parts of the dependency graph.  Here are some examples of
this pattern from the GHC build system:

\begin{itemize}
\item GHC contains Haskell packages which have metadata specified in
  \lst'.cabal' files. We need to extract the metadata and generate
  \lst'package-data.mk' files to be included into the build system; this
  is done by a Haskell program called \lst'ghc-cabal', which is built by the
  build system. Hence we do not know how to build the packages until we have
  built and run the \lst'ghc-cabal' tool.

  One ``solution'' to this problem is to generate the \lst'.mk'
  files and check them into the repository whenever they change, but
  checking in generated files is ugly and introduces more failure
  cases when the generated files get out of sync.

\item The dependencies of a Haskell source file can be extracted by
  running the compiler, much like using \lst'gcc -M' on a C source
  file.  But the Haskell compiler is one of the things that we are
  building, so we cannot know the dependencies of many of the Haskell
  source files until we have built the stage 1 compiler.

\item Source files processed by the C preprocessor using
  \lst'#include' directives to include other files.  We can only
  know what the dependencies are by running the C preprocessor to
  gather the set of filenames that were included.
\end{itemize}
\noindent
There are several other cases of such \emph{dynamic dependencies} in the build
system. Indeed, they arise naturally even when building a simple C program,
because the \lst'#include' files for a C source file are not known until the
source file is compiled or analysed by the C compiler, using something like
\lst'gcc -M'. Build systems that have a static dependency graph typically have
special-case support for changing dependencies; for example, Make will re-read
Makefiles that change during building.  In the case of Make,
this works for simple cases, but fails when there are dependencies
between the generated Makefiles, which arises in more complex cases.

The GHC build system works around this limitation of Make by
dividing the build system into \emph{phases}, where each phase
generates the parts of the build system that are required by
subsequent phases. We managed to reduce the number of phases to
three, by carefully making use of Make's automatic restarting
feature where possible. However, explicit phases are a terrible
solution for several reasons:

\begin{itemize}
\item Phases reduce concurrency: we cannot start the next phase until
  the previous one is complete.
\item Knowing what targets to put in each phase requires a deep
  understanding of generated parts of the build system and their dependencies,
  and this area of our build system is notoriously difficult to work on. 
  Diagnosing problems is particularly hard.
\item We have to run all the phases even when doing an incremental build,
 because explicit phases preclude dependency tracking across the phase boundary.
 The result is so slow that we were forced to allow the user to short-circuit
 the earlier phases, by promising that nothing has changed that would require
 rebuilding the early phases (\lst'make fast').  This solution is less than
 ideal.
\end{itemize}

Shake supports dynamic dependencies natively, so these problems just
go away.  The \lst"need" function can introduce new dependencies
\emph{after observing the results of previous dependencies}\footnote{It has been
suggested that the dependency mechanism in Shake should rightly be called
\emph{Monadic dependencies} to contrast with the \emph{Applicative dependencies}
in Make. We agree. In an applicative computation the structure cannot depend on
the values flowing through container. In a monadic computation the structure can
depend on the values.}
(see \S\ref{sec:need}). For example, when building a Haskell package, we first
\lst"need" the \lst'ghc-cabal' tool, then we run it on the package's \lst'.cabal' file to
extract the package metadata, and then we consult the result to determine what
needs to be done to build the package.

The existence of \lst"need" means that Shake cannot construct the
dependency graph ahead of time, because it doesn't know the full
dependency graph until the build has completed.  But this is not a
limitation in practice, whereas the lack of dynamic dependencies really
\emph{is} a limitation requiring painful workarounds.

% SimonM: I wasn't really sure what this paragraph was trying to say
% While the existence of the \lst"need" fundamentally alters the expressive
% power of Shake, for users it is merely a relaxation of an unnecessary rule
% (that \lst"need" must be the in a rule), so just makes things easier. At the
% build system level it requires many changes, the lack of a static dependency
% graph, termination checking as it executes, the ability to suspend partially
% executed rules etc. Fortunately, as users we do not need to care.

% \subsection{Fine-grain dependencies\label{sec:fine-grain-deps}}
%
% There are many instances where an action depends on a part of a file. As a
% compelling example in GHC, there are many configuration settings provided in a
% file \lst"system.config", which reads:
%
% \begin{lstlisting}
% system-ghc = /usr/bin/ghc
% system-gcc = /usr/bin/gcc
% ...
% \end{lstlisting}
%
% \vspace{-2mm}
% When executing a command we typically depend on only a handful of lines in this
% large configuration file. However, using simple file dependencies, we have to
% depend on the whole file. As a result, we end up rebuilding more than is
% necessary (see~\S\ref{sec:ghc}).
%
% One solution is to create a single file for each setting, but in GHC the whole
% \lst"system.config" file is generated by the \lst'configure' script, which makes
% this challenging. Since a single file is being generated, the next best solution
% is to chop it up into multiple small files, to provide accurate fine-grained
% dependencies.
%
% This approach does not work in Make. We can produce individual
% files, but the individual files must all depend on \lst"system.config", so they
% update appropriately. In the rule to generate the fragments, if we always generate the
% new file, then we do not break the dependencies -- everything still has an
% ultimate dependency on the whole file. If in a rule we chose to avoid generating
% the small file then the file always looks dirty (since its timestamp is older
% than that of \lst"system.config", which Make considers to signify dirty).
%
% Fortunately, Shake supports polymorphic dependencies that solve this problem,
% as discussed in~\S\ref{sec:polymorphic}. This brings dramatic improvements -- on
% real build systems we regularly see build times of a couple of seconds, which
% would have been minutes without this feature (e.g., see~\S\ref{sec:ghc}). This
% feature is also available in \lst'Tup' and \lst'Ninja', although in both cases
% requires explicitly enabling per rule.

% (using \lst'stat' in \lst'Ninja', and \texttt{\^} in Tup).

% Using the same approach in Shake we can simply avoid writing out the small file
% if it has not changed. Shake considers a file to be rebuilt if the rule runs
% successfully, but only considers the file \emph{changed} if its value changed.
% As a result, Shake can start down a build tree and then later avoid that build
% tree. This feature is built into the core of Shake, and means that, unlike
% Make, we can split a single dependency into fine-grained dependencies
% successfully.

% \subsection{Everything is a file}
%
% Once fine grained dependencies are available, they can quickly become
% pervasive. Using a build system based on filenames as keys and files as values
% we effectively reach the global store problem we criticised in \S? (particularly
% if we consider dyanmic dependencies from \S? to be \lst"$" dereferencing!).
% While not critical, Shake supports an Oracle feature, to allow us to make the
% keys and values arbitrary Haskell types, and store them in the main Shake
% database. Thanks to such a technique we can have an oracle per rule we create to
% track changing information, without necessarily doubling the number of files
% present in the build system. As an example:
%
% \begin{lstlisting}
% newtype ConfigKey = ConfigKey String
%     deriving (Show,Typeable,Eq,Hashable,Binary,NFData)
% rules = do
%     addOracle $ \(ConfigKey x) -> do
%         src <- readFile' "system.config"
%         Map.lookup (parseFile src) x
%
%     ... askOracle (ConfigKey x) ...
% \end{lstlisting}
%
% Here we define our own rule of type \lst"ConfigKey" which reads from the
% configuration and stores just that one field. Since this also participates in
% unchanging rules we automatically get rebuild avoidance if the value does not
% change.

\subsection{Avoiding external tools\label{sec:real_code}}

Build systems typically spend most of their time calling out to external
tools, e.g. compilers. But sometimes it is useful to replace external tools
with direct functions. As an example, the Make system used \lst'xargs' to
split command line arguments that exceed a certain size, but that did not work
consistently across OS versions, requiring lots of conditional logic.
Fortunately, with Haskell at our disposal, we can write:

\begin{lstlisting}
-- | @chunksOfSize size strings@ splits a given list of strings
-- into chunks not exceeding @size@ characters. If that is
-- impossible, it uses singleton chunks.
chunksOfSize :: Int -> [String] -> [[String]]
chunksOfSize n = repeatedly $ \xs ->
    let#\,#ys = takeWhile#\,#(<=#\,#n) $ scanl1#\,#(+) $ map#\,#length#\,#xs
    in splitAt (max 1 $ length ys) xs
\end{lstlisting}

\noindent Writing a small function is easy in Haskell, and even easier to test
(we test this function using QuickCheck). Writing it in \lst'bash' would be infeasible.
Reducing the specific behaviours required from external tools leads to
significantly less cross-platform concerns.

\subsection{Summary}

The unnecessary complexities described in this section have a big impact on
the overall complexity of the build system. The main lessons we have learnt are:

\begin{itemize}
\item Abstraction is a powerful and necessary tool. Lack of good abstraction
mechanisms is a significant cause of the complexity of previous attempts in
Make. Functional programming provides excellent abstractions --
marrying build systems and functional programming works well.
\item Expressive dependencies are important; we use
multiple outputs~\S\ref{sec:multiple-outputs},
resources~\S\ref{sec:ghc-pkg-db} and dynamic dependencies~\S\ref{sec:dynamic-deps}.
While some of these
features are only used in a few places (e.g. resources), their absence
requires pervasive workarounds, and a significant increase in the overall complexity.
\end{itemize}

% \subsection{Abstracting over patterns of dependencies}
%
% \todo{This was moved from Section 5.}
%
% By embedding the build system in Haskell we get access to its rich abstraction
% facilities. Examples:
% \begin{itemize}
%   \item Use general predicates instead of file patterns. For example, several
%   files are built by calling \texttt{genprimopcode}, which can be captured by a
%   single build OR-rule:
%
% \begin{lstlisting}
% [ "autogen/GHC/Prim.hs"
% , "GHC/PrimopWrappers.hs"
% , "*.hs-incl" ] |%> \file -> do ...
% \end{lstlisting}
%
%   \item Build rules with multiple outputs (AND-rules)
%   \item Compute build arguments from target filename
% \end{itemize}
