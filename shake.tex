\section{Background about Shake\label{sec:shake}}

Our solution to the issues raised in~\S\ref{sec:challenges} starts with
switching from Make to the Shake build system~\cite{shake}. In this section we
recap the key ideas behind Shake, and also describe some of the additional
features provided by Shake, but not covered in the original paper. These
features are of general use and all predate our efforts to replace the GHC build
system.

\subsection{Introduction\label{sec:shake_intro}}

As an example of a small Shake build system, let us compile a C file into an object file:

\begin{lstlisting}[numbers=left,xleftmargin=2em,framexleftmargin=1.5em]
module Main(main) where
import Development.Shake
import System.FilePath

main :: IO ()
main = shake shakeOptions $ do
    want ["foo.o"]

    "*.o" %> \out -> do
        let src = replaceExtension out "c"
        need [src]
        cmd "gcc -o" out "-c" src
\end{lstlisting}

\noindent Following the code from top to bottom:

\begin{description}
\item[Line 1] declares a Haskell module. Shake is a Haskell library, so all
Shake build systems are written in Haskell and can make full use of other
Haskell libraries and Haskell abstractions (functions, modules, packages,
\lst'let' expressions etc).
\item[Line 2] imports the \lst'Development.Shake' module, which provides most of
the functions and types in Shake. Some of the Shake API is given in
Figure~\ref{fig:shake-api}.
\item[Line 3] imports the \lst'System.FilePath' module, which in this example
provides the \lst'replaceExtension' function (we could have also used Shake's
convenient alias: \lst'let src = out -<.> "c"').
\item[Lines 6] declares the \lst'main' function, which calls \lst'shake'. The
\lst'shake' function takes some options (parallelism settings, etc.), along with
a set of \lst'Rules', and executes the necessary rules.
\item[Line 7] calls \lst'want' to declare that after the build system has
finished we would like the file \lst'foo.o' to be available and up-to-date.
\item[Line 9] defines a rule to build \lst'*.o' files, namely those files which
end with the extension \lst'.o'. The \lst'%>' operator produces a rule of type
\lst'Rules' which takes a pattern on the left, and an \lst'Action' on the
right. The variable \lst'out' will be bound to the actual file being produced,
namely \lst'foo.o' in this example.
\item[Line 10] computes the name of the source file, in our case
\lst'src = foo.c'.
\item[Line 11] uses \lst'need' to ensure \lst'foo.c' has been built before
continuing, and to introduce a dependency that if \lst'foo.c' changes then this
rule will require rerunning.
\item[Line 12] uses the variable-arity function \lst'cmd' to execute the system
command \lst'gcc' with appropriate arguments to produce \lst'out' from
\lst'src'. Since the \lst'Action' type has an instance of \lst'MonadIO' we can
do any \lst'IO' operation at this point.
\end{description}

\begin{figure}
\begin{lstlisting}
#\vspace{-7mm}#
#\line(1,0){240}#
newtype Rules a = ... #\hspace{36.5mm}\textbf{\emph{Generic API}}#
   deriving (Monoid, Functor, Applicative, Monad)
newtype Action a = ...
   deriving (Functor, Applicative, Monad, MonadIO)

data ShakeOptions = ShakeOptions {shakeThreads :: Int, ...}
shakeOptions :: ShakeOptions

shake :: ShakeOptions -> Rules () -> IO ()
action :: Action a -> Rules ()

type ShakeValue a =
    (Show#\,#a, Typeable#\,#a, Eq#\,#a, Hashable#\,#a, Binary#\,#a, NFData#\,#a)
data EqualCost = EqualCheap | EqualExpensive | NotEqual

class (ShakeValue key, ShakeValue value) =>
        Rule key value where
    storedValue :: ShakeOptions ->
        key -> IO (Maybe value)
    equalValue :: ShakeOptions ->
        key -> value -> value -> EqualCost

rule :: Rule key value =>
    (key -> Maybe (Action value)) -> Rules ()
apply :: Rule key value => [key] -> Action [value]

data Resource
newResource :: String -> Int -> Rules Resource
withResource :: Resource -> Int -> Action a -> Action a
#\vspace{-5mm}#
#\line(1,0){240}#
type FilePattern = String#\hspace{25mm}\textbf{\emph{File-specific API}}#
want :: [FilePath] -> Rules ()
need :: [FilePath] -> Action ()
needed :: [FilePath] -> Action ()
orderOnly :: [FilePath] -> Action ()
(%>) :: FilePattern -> (FilePath -> Action ()) -> Rules ()
(?>) :: (FilePath -> Bool)#\,#->#\,#(FilePath -> Action#\,#())#\,#->#\,#Rules#\,#()
\end{lstlisting}

\todo{\textbf{Neil}: Include everything we use in the paper}
\caption{Shake API\label{fig:shake-api}}
\end{figure}

% \todo{Include everything we use in the paper}
% \caption{Shake generic API\label{fig:shake_generic_api}}
% \end{figure}
% 
% \begin{figure}
% \begin{lstlisting}
% type FilePattern = String
% (%>) :: FilePattern -> (FilePath -> Action ()) -> Rules ()
% (?>) :: (FilePath -> Bool) -> (FilePath -> Action ()) -> Rules ()
% want :: [FilePath] -> Rules ()
% need :: [FilePath] -> Action ()
% needed :: [FilePath] -> Action ()
% orderOnly :: [FilePath] -> Action ()
% \end{lstlisting}
% \caption{Shake file-specific API\label{fig:shake_file_api}}
% \end{figure}

On the first execution, this example will start running the \lst'*.o' rule to
produce \lst'foo.o'. When execution gets to \lst'need [src]' this rule will stop
and the rule for \lst'foo.c' will be run. Shake provides a default rule for
files that do not match any other rules, which simply checks the file already
exists. After completing this simple rule, Shake will resume running the
\lst'*.o' rule, executing \lst'gcc' to build \lst'foo.o'.

The major source of Shake's expressive power is that in Shake a \lst'need' can
come at any point during a rule's \lst'Action'. In particular, we can perform
arbitrary computation including I/O, running system commands and examining files
produced by a previous \lst'need' before deciding which future \lst'need' to
perform. In contrast, in most build systems, all dependencies of a rule must be
given before the rule starts executing. We make extensive use of this additional
power throughout \S\ref{sec:abstractions}.

\subsection{Parsimonious rebuilding}

Both Make and Shake try to eliminate unnecessary rebuilding, but both use
different approaches:

\begin{itemize}
\item In Make, a rule is rerun if the output does not exist, or if the
modification time of the output is earlier than any of the inputs. In an example
similar to that above, \lst'gcc' would be rerun if either \lst'foo.o' did not
exist, or if the modification time of \lst'foo.o' was earlier than that of
\lst'foo.c'.
\item In Shake, a rule is rerun if any of the inputs or outputs change. In our
example, Shake will rerun \lst'gcc' if either \lst'foo.c' or \lst'foo.o' does
not exist or if either file changes modification time from when the rule was
last run (or more generally, if either file changes contents, see
\S\ref{sec:file-contents}).
\end{itemize}

Shake achieves this result by storing the inputs and outputs after a rule
completes in a per-project database, and rerunning a rule if anything has
changed (and thus the result could be expected to change). The Shake's approach
has a number of advantages: 1) we can track things that aren't files as in
\S\ref{sec:polymorphic}; 2) we can depend on a more refined measure than
modification time as in \S\ref{sec:file-contents}; 3) we are robust to system
time changes or extracting old files from backups; 4) the execution of a rule
does not need to update its output to avoid future rebuilds. The final point is
often worked around in Make by using \emph{stamp files} to provide a modification
time but having no actual content -- Shake simplifies this common use-case.

\subsection{Post-use dependencies}

Looking at our first example, the object gets recompiled if the C source file
changes. But C files may \lst'#include' any number of header files, and changes
to these headers will also affect the resulting object. While it is possible to
write our own code to search for transitively included headers
\cite[\S6.4]{shake}, a simpler alternative is to reuse the logic already present
in \lst'gcc'. The \lst'gcc' compilation command (\lst'-c') can also take a flag
\lst'-M' to print out a \lst'Makefile' listing all headers that were used by the
compilation. We can integrate this into our example by replacing the final
line with:

\begin{lstlisting}
Stdout makefile <- cmd "gcc -M -o" out "-c" src
need $ makefileDepends makefile
\end{lstlisting}

\noindent We use \lst'Stdout' to capture the output of \lst'cmd', then
\lst'makefileDepends' to parse the \lst'Makefile' and return all dependencies
\footnote{Using the \lst'parseMakefile' helper function in Shake we can define
\lst'makefileDepends = concatMap snd . parseMakefile'.}. After obtaining the
list of dependencies we \lst'need' them, making them inputs to this rule, and
ensuring that if a header changes the object will be rebuilt.

However, for this use case, the \lst'need' function is actually a little
\emph{too} powerful. The \lst'need' will cause Shake to suspend the \lst'*.o'
rule, build the dependencies, and then resume the remainder of the rule. But
since the header files were \emph{already} used by \lst'gcc', if they change
then the result will be inconsistent, since the object file will reflect the
previous header contents.

We solve this problem by using \lst'needed' instead of \lst'need', which
combines \lst'need' which an assertion that the file does not change as a result
of building, and thus the result is consistent. In the common case of the header
files being source files, the associated rule will be the default file rule,
which does not modify the file, and the assertion will not trigger.

\subsection{Order-only dependencies}

We have seen how \lst'needed' can be safely used for source files, but what
about generated files? Imagine we have a rule to build \lst'config.h' from a
configuration data file. With the existing formulation,
\lst'needed ["config.h"]' will raise an error if the configuration data has
changed and \lst'config.h' is not up-to-date. One solution is to \lst'need ["config.h"]'
\emph{before} executing \lst'gcc'. This solution is safe -- the file
\lst'config.h' will always be built before it is used and will not change when
\lst'needed' (since it has already been built). However, this solution
introduces a dependency on \lst'config.h', which causes any object file that
does \emph{not} include \lst'config.h' to rebuild unnecessarily if
\lst'config.h' changes.

A better solution is to use order-only dependencies, with the expression
\lst'orderOnly ["config.h"]'. This expression ensures that \lst'config.h' has
been built and is up-to-date before continuing, but does not introduce a
dependency on \lst'config.h'. Afterwards, if the file turns out to have been
required, \lst'needed' can be used to express the dependency.

Combining \lst'orderOnly' (ensure a file has been built) and \lst'needed' (depend on a file) we obtain the equivalence:

\begin{lstlisting}
need xs === (orderOnly xs >> needed xs)
\end{lstlisting}

\noindent Although in practice both \lst'orderOnly' and \lst'needed' can be
implemented on top of the \lst'need' primitive.

\subsection{Polymorphic dependencies\label{sec:polymorphic}}

While build-systems are typically focused around files, the core of Shake is
fully polymorphic, operating on objects that are uniquely labelled by a
\lst'key' (e.g. filename) and have rules that produce a \lst'value' (e.g.
modification time). This polymorphic API and the file-specific API built on top
of it are presented in corresponding sections of Figure~\ref{fig:shake-api}. Using
the polymorphic API we can define new types of rules to track additional
dependencies.

As an example of a new type of rules, let us imagine that the actual compiler
name (\lst'gcc') is obtained from a configure script that produces a
configuration file containing many key/value pairs. The \lst'*.o' rule could
depend on and parse the configuration file, but then any change to the
configuration file would cause all rules using the C compiler to rebuild, even
if the compiler name did not change. Alternatively, we could have rules to split
the single configuration file into one file per key, where each file is only
updated if that key changes, but that might result in thousands of files.
Instead, we can define a rule to map from configuration keys to values.

\begin{lstlisting}
newtype ConfigKey = ConfigKey String deriving
    (Show#\,#a, Typeable#\,#a, Eq#\,#a, Hashable#\,#a, Binary#\,#a, NFData#\,#a)

instance Rule ConfigKey String where
    storedValue _ _ = return Nothing

main = shake shakeOptions $ do
    rule $ \(ConfigKey key) -> Just $ do
        src <- readFile' "config.file"
        return $ parseConfig src Map.! key

    "*.o" %> \out -> do
        cc <- apply [ConfigKey "cc"]
        ...
        cmd cc "-o" out "-c" src

parseConfig :: String -> Map String String
\end{lstlisting}

This code defines a fresh type \lst'ConfigKey' to serve as the key for the rule.
We define a \lst'Rule' instance so Shake rules can operate on \lst'ConfigKey'
keys, declaring the associated value to be of type \lst'String'. We are only
required to implement the \lst'storedValue' method, where we declare that the
rule producing the config key is not stored separately on disk.

Next we use \lst'rule' to describe how to generate the value associated with
\lst'ConfigKey', using \lst"readFile'" to read the config file, and assuming the
existence of a suitable \lst'parseConfig' function. We use \lst'apply' to force
execution of the rule and introduce a dependency on the result.
%For files, \lst'%>' is implemented in terms of \lst'rule', and \lst'need' is
% implemented in terms of \lst'apply'.
Now the \lst'*.o' rule will have the \emph{value} of the
\lst'cc' key as a dependency, and if that value changes, all \lst'*.o' files will rebuild.

% \todo{Do we need to explain oracles? Just polymorphic rules aren't that much harder}
% 
% The use of non-file rule types is common in large build systems, occurring ten times in our implementation of GHC's build system in \S\ref{sec:ghc}. To simplify such uses we define an oracle abstraction, allowing us to write:
% 
% \begin{lstlisting}
% configKey <- addOracle $ \(ConfigKey key) -> do
%     src <- readFile' "config.file"
%     return $ parseConfig src Map.! key
% 
% "*.o" %> \out -> do
%     cc <- configKey $ ConfigKey key
%     ...
% \end{lstlisting}
% 
% Using oracles we are freed from defining our own \lst'Rule' instance and can use a type-safe version of \lst'apply' tailored to our particular oracle. The definition of oracles is only a handful of lines long, but sharply decreases the number of Shake concepts that must be understood to define non-file rules following the typical pattern. All non-file rules in GHC are implemented using oracles.

The use of non-file rule types is common in large build systems, occurring ten times in our implementation of GHC's build system in \S\ref{sec:ghc}. Polymorphic dependencies do not give any fundamental additional expressive power over file rules, but they do provide:

\begin{itemize}
\item An easy way to get unique keys, by allowing use of a freshly defined type. While two unrelated parts of the build system might both define \lst'config-key-cc.txt' rules and clash, any two definitions of \lst'ConfigKey' will either be distinct via their qualified name, or will result in a compile error.
\item Additional structure for keys and values, instead of hierarchical filenames and binary content files. In particular keys can contain full algebraic data types including lists, tuples and booleans.
\item Greater granularity, by not forcing each rule to be backed by a separate file. GHC's build system has 6,677 non-file rules.
\end{itemize}
 
\subsection{Limiting Parallelism}

Like most build systems, Shake can run multiple rules in parallel. However, running too many rules in parallel can cause a computer to become overloaded, spending much of its time switching between tasks instead of executing them. Most build systems take an argument to set the maximum number of rules that can be run simultaneously (Shake has \lst'shakeThreads'), and typically that number is based on the number of available CPUs. However, there are other instances where certain rules should not be run in parallel:

\begin{enumerate}
\item Some APIs are global in nature. If you run two programs that access the Excel API simultaneously things start to fail.
\item Many people have large numbers of CPUs, but only one slow rotating hard-drive. Running many disk-heavy linker processes simultaneously can overload the hard-drive.
\item Some proprietary software has licenses which limit the number of concurrent processes, for example ModelSim.
\end{enumerate}

Build systems typically obey such constraints using one of the following approaches:

\begin{enumerate}
\item Limit the maximum number of simultaneous rules. For example a user with 24 CPUs might limit the build system to 8 simultaneous rules so that linking is not problematic. This solution wastes CPU resources, leaving CPUs idling.
\item Add locks to suspend jobs that are competing for a shared resource. For example any rule using the Excel API could check for the existence of a lock file before continuing, and keep a lock file while executing. This approach still wastes CPU resources, causing them to block instead of executing other rules.
\item Use fake dependencies to serialise the rules into a linear order, ensuring only one rule using the resource can run at a time. This approach reduces available parallelism and requires a global ordering on the build rules.
\end{enumerate}

In Shake this problem can be solved directly using the \lst'Resource' type, which represents a finite resource that multiple build rules can use. Resource values are created with \lst'newResource' and used by \lst'withResource'. As an example, only one set of calls to the Excel API can occur at once, therefore Excel is a finite resource of quantity 1. We can write:

\begin{lstlisting}
want ["a.xls","b.xls"]
excel <- newResource "Excel" 1
"*.xls" %> \out ->
    withResource excel 1 $ cmd "excel" out ...
\end{lstlisting}

We create a new resource named \lst'excel' of quantity one, then when using Excel we call \lst'withResource excel 1', to indicate we use one Excel resource. Now we will never run two copies of Excel simultaneously, regardless of the \lst'shakeThreads' setting. Moreover, the build system will never block waiting for the resource if there are other rules that could be run. We use this approach to deal with the \lst'ghc-pkg' tool, which only permits one writer or multiple readers, see \S?.

\subsection{Tracking File Contents\label{sec:file-contents}}

Build systems run actions on files, skipping the actions if the files have not changed. An important part of that process involves determining if a file has changed. As we saw in \S\ref{sec:shake_intro}, \make{} uses modification time to check outputs are newer than inputs, while Shake uses modification time as a proxy for file contents and rebuilds on any change. However, there are two common cases where a file can change modification time without changing its contents:

\begin{itemize}
\item When running a deterministic program that consumes a textual format, there are often many changes that can be made to the input without changing the generated output. For example, when generating a C source file from an XML input file, whitespace around XML tags is typically ignored as are XML comments. In such instances the generator can arrange to avoid rewriting the output file, or the build rule can write to a separate location and copy the file over if there are differences. However, such accommodation needs to be made for many actions.
\item When working on two \texttt{git} branches, both of which are based on a common \texttt{master} branch, a typical pattern is to switch from one branch to another. If the first branch was recently synced with \texttt{master}, but the second has not been for a while, the typical workflow is to switch to the second branch and then merge with \texttt{master}. Assuming the differences between the two branches are small, the number of changed files is also likely to be small. However, if \texttt{master} changes regularly, any files that changed in \texttt{master} since the second branch was last synced will have a new modification time, despite having the same contents.
\end{itemize}

Shake could solve these problems by simply making the value of a file rule the contents of that file, instead of the modification time. In the remainder of this section we discuss how to efficiently implement something approximating that simple scheme. Since Shake rules are polymorphic it is easy to provide multiple types of coexisting file rules, although for simplicity we instead provide the option of which file value type to use as a field of \lst'ShakeOptions'.

The obvious problem with simply storing the entire contents of all files is that it will result in a huge Shake database. Instead, we store the result of applying a hash function to the file contents, where the hash changing indicates the file contents have changed. There is a remote risk that the file will change without its hash changing, but unless the build system users are actively hostile, that is unlikely. The disadvantage of content hashes over modification times is that hashes are expensive to compute, requiring a full scan of the file. In particular, after a rule finishes Shake must scan the file it just built, and on startup Shake must scan all files. Scanning all files can cause rebuilds where no actions are required to take minutes instead of less than a second.

As an optimisation, Shake stores the modification time, file size and hash of the contents. After a rule completes all the information is computed and stored. When checking if a file has changed, first the modification time is checked, and if that matches, the contents are assumed to have not changed. If the modification time has changed, and the file size has also changed, then the file has definitely changed. Only in the case where the modification time has changed but the size has not do we compute the actual hash. If that hash is equal to the previously recorded hash we store a new modification time, so that future checks will avoid computing the hash. These optimisations give most of the benefits of storing the file contents, but with significantly reduced costs.
